\documentclass[10pt,a4paper]{article}
\usepackage{fullpage}
\usepackage{graphicx}
\usepackage{fancyhdr}
\usepackage{occi}
\setlength{\headheight}{13pt}
\pagestyle{fancy}

% default sans-serif
\renewcommand{\familydefault}{\sfdefault}

% no lines for headers and footers
\renewcommand{\headrulewidth}{0pt}
\renewcommand{\footrulewidth}{0pt}

% header
\fancyhf{}
\lhead{GWD-R}
\rhead{\today}

% footer
\lfoot{occi-wg@ogf.org}
\rfoot{\thepage}

% paragraphs need some space...
\setlength{\parindent}{0pt}
\setlength{\parskip}{1ex plus 0.5ex minus 0.2ex}

% some space between header and text...
\headsep 13pt

\setcounter{secnumdepth}{4}

\begin{document}

% header on first page is different
\thispagestyle{empty}

GWD-R \hfill Thijs Metsch, Platform Computing\\
OCCI-WG \hfill Andy Edmonds, Intel\\
\rightline {October 7, 2010}\\
\rightline {Updated: \today}

\vspace*{0.5in}

\begin{Large}
\textbf{Open Cloud Computing Interface - RESTful HTTP Rendering}
\end{Large}

\vspace*{0.5in}

\underline{Status of this Document}

This document provides information to the community regarding the
specification of the Open Cloud Computing Interface. Distribution is
unlimited.

\underline{Copyright Notice}

Copyright \copyright Open Grid Forum (2009-2010). All Rights Reserved.

\underline{Trademarks}

OCCI is a trademark of the Open Grid Forum.

\underline{Abstract}

This document, part of a document series, produced by the OCCI working
group within the Open Grid Forum (OGF), provides a high-level
definition of a Protocol and API. The document is based upon
previously gathered requirements and focuses on the scope of important
capabilities required to support modern service offerings.

\newpage
\tableofcontents
\newpage

\section{Introduction}
The Open Cloud Computing Interface (OCCI) is a RESTful Protocol and
API for all kinds of management tasks. OCCI was originally initiated
to create a remote management API for IaaS%
\footnote{Infrastructure as a Service}
model-based services, allowing for the development of interoperable tools for
common tasks including deployment, autonomic scaling and monitoring.
%
It has since evolved into an flexible API with a strong focus on
interoperability while still offering a high degree of extensibility. The
current release of the Open Cloud Computing Interface is suitable to serve many
other models in addition to IaaS, including e.g.~PaaS and SaaS.

In order to be modular and extensible the current OCCI specification is
released as a suite of complimentary documents, which together form the complete
specification.
%
The documents are divided into three categories consisting of the OCCI Core,
the OCCI Renderings and the OCCI Extensions.
%
\begin{itemize}
\item The OCCI Core specification consists of a single document defining the
 OCCI Core Model. The OCCI Core Model can be interacted with {\em
 renderings} (including associated behaviours) and expanded through {\em extensions}.
\item The OCCI Rendering specifications consist of multiple documents each
 describing a particular rendering of the OCCI Core Model. Multiple renderings can
 interact with the same instance of the OCCI Core Model and will automatically support
 any additions to the model which follow the extension rules defined in OCCI
 Core.
\item The OCCI Extension specifications consist of multiple documents each
 describing a particular extension of the OCCI Core Model. The extension documents
 describe additions to the OCCI Core Model defined within the OCCI specification
 suite.
\end{itemize}
%
The current specification consist of three documents.
Future releases of OCCI may include additional rendering and extension
specifications. The documents of the current OCCI specification suite are:

\begin{description}
\item[OCCI Core] describes the formal definition of the the OCCI Core Model
\cite{occi:core}.
\item[OCCI HTTP Rendering] defines how to interact with the OCCI Core Model using the
RESTful OCCI API \cite{occi:http_rendering}. The document defines how the OCCI Core Model can
be communicated and thus serialised using the HTTP protocol.
\item[OCCI Infrastructure] contains the definition of the OCCI Infrastructure
extension for the IaaS domain \cite{occi:infrastructure}. The document defines
additional resource types, their attributes and the actions that can be taken
on each resource type.
\end{description}


\section{Notational Conventions}
All these parts and the information within are mandatory for
implementors (unless otherwise specified). The key words "MUST", "MUST
NOT", "REQUIRED", "SHALL", "SHALL NOT", "SHOULD", "SHOULD NOT",
"RECOMMENDED", "MAY", and "OPTIONAL" in this document are to be
interpreted as described in RFC 2119 \cite{rfc2119}.

\textbf{Andy: we need to state that this document as part of the current document set,
supersedes all previous documents.}


All examples in this document use one of the following three
Categories. An example name-space is also given. Syntax and Semantics
is explained in the remaining sections of the document. These examples
do not strive to be complete but to show the functionalities OCCI has:

\begin{verbatim}

Category:compute;scheme=http://schemas.ogf.org/occi/infrastructure;location=/compute 
                                                           (This is an compute kind)
Category:storage;scheme=http://schemas.ogf.org/occi/infrastructure;location=/storage 
                                                           (This is an storage kind)
Category:my_stuff;scheme=http://example.com/occi/my_stuff;location=/my_stuff 
                                                           (This is a mixin of user1)

The following namespace hierarchy is used in the examples:

http://example.com/-/
http://example.com/vms/user1/vm1
http://example.com/vms/user1/vm2
http://example.com/vms/user2/vm1
http://example.com/disks/user1/disk1
http://example.com/disks/user2/disk1
http://example.com/compute/
http://example.com/storage/
http://example.com/my_stuff/
\end{verbatim}

\section{A RESTful HTTP rendering for OCCI compliant Interfaces}
The HTTP Protocol is the underlying core fabric of OCCI and uses all
the features of the HTTP and underlying protocols (like self-healing
capabilities of TCP) offer. OCCI also builds upon the Resource
Oriented Architecture (ROA). ROA's use Representation State Transfer
(REST) to cater for client and service interactions. Interaction with
the system is by inspection and modification of a set of related
resources and their states, be it on the complete state or a
sub-set. Resources must be uniquely identified. HTTP is an ideal
protocol to use in ROA systems as it provides the means to uniquely
identify individual resources through URLs as well as operating upon
them with a set of general-purpose HTTP verbs. These HTTP verbs map
loosely to the resource related operations of Create (POST), Retrieve
(GET), Update (POST/PUT) and Delete (DELETE).

The following notations are used when referring to parts or complete
URIs:

\begin{verbatim}
http://www.example.com:8080/foo/bar;action=stop
<   >  <     Authority    >< Path >< Fragment >
  ^
  Scheme
\end{verbatim}

The following section describe the general behavior for all HTTP based
renderings. Later sections will describe the syntax and semantic of
how to render the OCCI Core model with different Content-Types.

Each Kind sub-type instance within an OCCI system must be uniquely
identified by an URI. The structure of these URIs is opaque and the
system should not assume a static, pre-determined scheme for their
structure (Example: \emph{http://example.com/vms/user1/vm1}).

\subsection{Behavior of the HTTP Verbs}
As OCCI adopts a ROA, REST-based architecture and uses HTTP as the
foundation protocol the means of interaction with all RESTful resource
instances is through the four main HTTP verbs. OCCI service
implementations must, at a minimum, support these verbs:

\textbf{TODO: add table}

\subsection{A RESTful rendering of OCCI}
The following sections and paragraphs described how the OCCI model
MUST be implemented by service providers. Operations which are not
defined are out of scope for this specification and MAY be implemented
by the service provider. This is just a minimal set to ensure
interoperability.

\subsubsection{Name-space Hierarchy and locations}
The name-space and the hierarchy are free definable by the Service
Provider. Although the Service Provider MUST implement the location
path feature which is required by OCCI for discovery capabilities and
operations on \hl{Mixin}s and \hl{Kind}s. Location paths tell the
client where all resource instance of one \hl{Kind} or \hl{Mixin} can
be found regardless of the hierarchy the service provider
defines. These paths are discoverable by the client through the Query
interface.

These location paths can be part of the name-space or rendered
alongside. The following example shows how the locations paths are
rendered alongside the name-space hierarchy:

\begin{verbatim}
Category:compute;scheme=http://schemas.ogf.org/occi/infrastructure;location=/compute 
                                                           (This is an compute kind)
Category:storage;scheme=http://schemas.ogf.org/occi/infrastructure;location=/storage 
                                                           (This is an storage kind)
Category:my_stuff;scheme=http://example.com/occi/my_stuff;location=/my_stuff 
                                                           (This is a mixin of user1)

The following namespace hierarchy is used in the examples:

http://example.com/-/
http://example.com/vms/user1/vm1
http://example.com/vms/user1/vm2
http://example.com/vms/user2/vm1
http://example.com/disks/user1/disk1
http://example.com/disks/user2/disk1
http://example.com/compute/
http://example.com/storage/
http://example.com/my_stuff/
\end{verbatim}

Location paths can also be part of the name-space
hierarchy:\footnote{/vms/user1/vm1 (= OCCI base type ID) is a resource
  instance below the name-space path /vms/user1/.}

\begin{verbatim}
Category:compute;scheme=http://schemas.ogf.org/occi/infrastructure;location=/vms 
                                                           (This is an compute kind)
Category:storage;scheme=http://schemas.ogf.org/occi/infrastructure;location=/disks 
                                                           (This is an storage kind)
Category:my_stuff;scheme=http://example.com/occi/my_stuff;location=/my_stuff 
                                                           (This is a mixin of user1)

The following namespace hierarchy is used in the examples:

http://example.com/-/
http://example.com/vms/user1/vm1
http://example.com/vms/user1/vm2
http://example.com/vms/user2/vm1
http://example.com/disks/user1/disk1
http://example.com/disks/user2/disk1
http://example.com/my_stuff/
\end{verbatim}

\subsubsection{Various Operations and their prerequisites and behaviors}

\paragraph{Operations on resource instances}
The following operations MUST be implemented by the service provider
for operations of resource instances. The resource instance is
uniquily identified by an URI (For example:
\emph{http://example.com/vms/user1/vm1} - Note: the path MUST not end
with an '/').

\begin{description}
\item[Creating a resource instance] A request to create a resource
  instance MUST contain at least one Category definition which is (or
  relates to) a \hl{Kind} definition. If multiple Categories are
  defined the first one which is (or relates to) to a \hl{Kind} MUST
  be used for defining the type of the resource instance. Optional
  information which might be provided by the client and if available
  MUST be used are Links and Attributes. Two ways can be used to
  create a new resource instance - HTTP POST or PUT:
\begin{verbatim}
> POST / HTTP/1.1
> [...]
> 
> Category:compute;scheme=http://schemas.ogf.org/occi/infrastructure
> X-OCCI-Attribute:occi.compute.cores=2 occi.compute.hostname=foobar
> [...]
 
< HTTP/1.1 200 OK
< [...]
< Location:http://example.com/vms/user1/vm1
\end{verbatim}
  The path on which this POST verb is executed can be any existing
  path in the hierachy of the service providers name-space. The
  service must return the Location of the newly created resource
  instance.

  Or HTTP PUT can be used. In this case the client ask the service
  provider to create a resource instance at a certain path in the
  namespace hierarchy.\footnote{If a Service Provider does not want the
    user to define the path of a resource instance it can return a Bad
    Request return code - See section \ref{sec:return_codes}.}
\begin{verbatim}
> PUT /vms/user1/my_first_virtual_machine HTTP/1.1
> [...]
> 
> Category:compute;scheme=http://schemas.ogf.org/occi/infrastructure
> X-OCCI-Attribute:occi.compute.cores=2 occi.compute.hostname=foobar
> [...]
 
< HTTP/1.1 200 OK
< [...]
\end{verbatim}
  The service will return an OK code.

\item[Retrieving a resource instance] For retrieval the HTTP GET verb
  is used. It MUST return at least the Category which defines the
  \hl{Kind} of the resource instance. Links pointing to references
  resource instances, other URI or Actions (This SHOULD only be the
  \hl{Actions} which are currently applicable) MUST be included if
  present. The Attributes of the resource instance MUST be exposed to
  the client if available.
\begin{verbatim}
> GET /vms/user1/vm1 HTTP/1.1
> [...]
 
< HTTP/1.1 200 OK
< [...]
< Category:compute;scheme=http://schemas.ogf.org/occi/infrastructure
< Category:my_stuff;scheme=http://example.com/occi/my_stuff
< X-OCCI-Attribute:occi.compute.cores=2 occi.compute.hostname=foobar
< Link: [...]
\end{verbatim}

\item[Updating a resource instance] Before updating a resource
  instance it is RECOMMENDED that the client first retrieves the
  resource instance. Updating is done using the HTTP PUT verb. Only
  the information (Links, Attributes or \hl{Mixin} Categories) which
  is updated MUST be provided along with the
  request.\footnote{Changing the type of the resource instance MUST
    not be possible.}
\begin{verbatim}
> PUT /vms/user1/vm1 HTTP/1.1
> [...]
> 
> X-OCCI-Attribute:occi.compute.memory=4G
> [...]
 
< HTTP/1.1 200 OK
< [...]
\end{verbatim}

\item[Deleting a resource instance] A resource instance can be deleted
  using the HTTP DELETE verb. No other information SHOULD be added to
  the request
\begin{verbatim}
> DELETE /vms/user1/vm1 HTTP/1.1
> [...]

< HTTP/1.1 200 OK
< [...]
\end{verbatim}

\item[Triggering an action on a resource instance] To trigger an
  action on a resource instance the request MUST containg the Category
  defining the \hl{Action}. It MAY include attributes which are the
  parameters of the action. Actions are triggered using the HTTP POST
  verb and by adding a fragment to the url. This fragment exposes the
  term of the \hl{Action}. If an action is not available or a proper
  return code should be returned.
\begin{verbatim}
> POST /vms/user1/vm1;action=stop HTTP/1.1
> [...]
> X-OCCI-Attribute:flag=poweroff

< HTTP/1.1 200 OK
< [...]
\end{verbatim}
\end{description}

\paragraph{Handling the Query interface}
The query interface MUST be implemented by all services supporting
OCCI. It MUST be found at the path \emph{/-/}. The following
operations MUST be implemented by the service:

\begin{description}
\item[Retrieving all registered Categories] The HTTP verb GET must be
  used to retrieve all categories the service can handle. This allows
  the client to discover the capabilites of the Service provider. The
  result MUST contain all information about the Category (including
  Attributes and Actions assigned to this Category).
\begin{verbatim}
> GET /-/ HTTP/1.1
> [...]
 
< HTTP/1.1 200 OK
< [...]
< Category:compute;scheme=http://schemas.ogf.org/occi/infrastructure; \
                   attributes=occi.compute.cores,occi...; \
                   rel=http://schemas.ogf.org/occi/core\#entity; \
                   actions=http://schemas.ogf.org/occi/infrastructure/compute/action#stop,...; \
                   location=/compute
< Category:my_stuff;scheme=http://example.com/occi/my_stuff; \
                    location=/my_stuff
< Category:storage;scheme=http://schemas.ogf.org/occi/infrastructure; \
                   attributes=...; \
                   actions=...; \
                   rel=http://schemas.ogf.org/occi/core\#entity; \
                   location=/storage
< 
\end{verbatim}
\textbf{Note:} A Service provider SHOULD support a filtering
mechanism. If a Category is provided in the request the server SHOULD
only return the complete rendering of the provided Category.

\item[Adding a \hl{Mixin} category defintion] To add a \hl{Mixin} to
  the service the HTTP PUT verb MUST be used. All possible information
  for the \hl{Mixin} category must be defined. At least the Category
  term, scheme and location MUST be defined. Actions and Attributes
  are optional:
\begin{verbatim}
> GET /-/ HTTP/1.1
> [...]
> Category:my_stuff;scheme=http://example.com/occi/my_stuff; \
                    rel=http:/example.com/occi/something_else#mixin; \
                    attributes=...; \
                    actions=...; \
                    location=/my_stuff

< HTTP/1.1 200 OK
< [...]
\end{verbatim}
The service might reject this request if it does not allow
user-defined Categories to be created. Also on name collesions of the
defined location path the service provider might reject this
operation.

\item[Removing a \hl{Mixin} category definition] A user defined
  \hl{Mixin} CAN be removed (if allowed) by using the HTTP DELETE
  verb. The information about wich \hl{Mixin} should be deleted MUST
  be provided in the request:
\begin{verbatim}
> DELETE /-/ HTTP/1.1
> [...]
> Category:my_stuff;scheme=http://example.com/occi/my_stuff;

< HTTP/1.1 200 OK
< [...]
\end{verbatim}
\end{description}

\paragraph{Operations on \hl{Mixin}s or \hl{Kind}s}
All the following operations CAN only be be done on a location path
provided by category definition. It MUST end with an \emph{/}.

\begin{description}
\item[Retrieving all resource instances belonging to \hl{Mixin} or
  \hl{Kind}] The HTTP verb GET must be used to retrieve all resource
  instances. The service provider MUST return a URI-list containing
  all resource instances which belong to the requested \hl{Mixin} or
  \hl{Kind}:
\begin{verbatim}
> GET /compute/ HTTP/1.1
> [...]
 
< HTTP/1.1 200 OK
< Content-type: text/uri-list
< [...]
< 
< http://example.com/vms/user1/vm1
< http://example.com/vms/user1/vm2
< http://example.com/vms/user2/vm1
\end{verbatim}
\textbf{Note:} A Service provider SHOULD support a filtering
mechanism. If a category is provided in the request the server SHOULD
only return the resource instances belonging to the provided
Category. The provided category definition SHOULD be different from
the one Category definition which defined the location path used in
the request.

\item[Triggering actions on all instances of a \hl{Mixin} or
  \hl{Kind}] Actions can be triggered on all resource instances of the
  same \hl{Mixin} or \hl{Kind}. The HTTP POST verb MUST be used. Also
  the action MUST be defined by the category which defines the
  location path which is used in the request:
\begin{verbatim}
> POST /compute/;action=stop HTTP/1.1
> [...]
> X-OCCI-Attribute:flag=poweroff

< HTTP/1.1 200 OK
< [...]
\end{verbatim}

\item[Adding a resource instance to a \hl{Mixin}] One or multiple
  resource instances can be added to a \hl{Mixin} using the HTTP PUT
  verb. The URIs which uniquely defined the resource instance MUST be
  provided in the request:
\begin{verbatim}
> PUT /my_stuff/ HTTP/1.1
> [...]
> X-OCCI-Location:http://example.com/vms/user1/vm1, \
                  http://example.com/vms/user1/vm2, \
                  http://example.com/disks/user1/disk1

< HTTP/1.1 200 OK
< [...]
\end{verbatim}

\item[Removing a resource instance from a \hl{Mixin}] One or multiple
  resource instances can be removed from a \hl{Mixin} using the HTTP
  DELETE verb. The URIs which uniquely defined the resource instance
  MUST be provided in the request:
\begin{verbatim}
> DELETE /my_stuff/ HTTP/1.1
> [...]
> X-OCCI-Location:http://example.com/vms/user1/vm1, \
                  http://example.com/vms/user1/vm2, \
                  http://example.com/disks/user1/disk1

< HTTP/1.1 200 OK
< [...]
\end{verbatim}
\end{description}

\paragraph{Operation on paths in the namespace}
The following operations are defined when operating on paths in the
name-space hirachy which are not location paths nor resource
instances. They MUST end with \emph{/} (For example
\emph{http://example.com/vms/user1}).

\begin{description}
\item[Retrieving all resource instances below a path] The HTTP verb
  GET must be used to retrieve all resource instances. The service
  provider MUST return a URI-list containing all resource instances
  which are childs of the provided URI in the name-space hierarchy:
\begin{verbatim}
> GET /vms/user1/ HTTP/1.1
> [...]
 
< HTTP/1.1 200 OK
< Content-type: text/uri-list
< [...]
< 
< http://example.com/vms/user1/vm1
< http://example.com/vms/user1/vm2
\end{verbatim}
\textbf{Note:} A Service provider SHOULD support a filtering
mechanism. If a category is provided in the request the server SHOULD
only return the resource instances belonging to the provided
Category.

\item[Deletetion all resource instances below a path] The HTTP verb
  DELETE must be used to delete all resource instances under a
  hierarchy:
\begin{verbatim}
> DELETE /vms/user1/ HTTP/1.1
> [...]
 
< HTTP/1.1 200 OK
< [...]
\end{verbatim}
\textbf{Note:} A Service provider SHOULD support a filtering
mechanism. If a category is provided in the request the server SHOULD
only return the resource instances belonging to the provided
Category.
\end{description}

\subsection{Syntax and semantics of the rendering}
\label{sec:syntax}
The following subsections demonstrate how the OCCI base types can be
syntactically rendered.

\subsubsection{Rendering of an OCCI-Category}

The semantics of the Category in the OCCI context is described in the
OCCI Core \& Models document.\footnote{This rendering follows the
  Category header defined by the Web Categories specification,
  http://tools.ietf.org/html/draft-johnston-http-category-header-01
  and MUST be rendered accordingly.}

\begin{verbatim}
Category: <term>; scheme="<scheme>"
    [;rel=<space-separated list of related Category identifiers>]
    [;attributes=<space-seperated list of attribute names>]
    [;title=<Title of this Category>]    
    [;location=<Parent location>]
\end{verbatim}
There is NO order for the optional part.

\subsubsection{Rendering of OCCI-Links and OCCI-Actions}
The semantics of the Link header in the OCCI context is described in
the OCCI Core \& Models document.\footnote{This rendering follows the
  Link header defined by the Web Linking specification,
  http://tools.ietf.org/html/draft-nottingham-http-link-header-10 and
  MUST be rendered accordingly.}

\begin{verbatim}
Link: <Resource URL>;
    rel=<space-separated list of Category identifiers of the target Resource type>
    [;self=<Link instance URL>]
    [;category=<space-separated list of Category identifiers of the Link type>
    [;<attribute name>=<attribute value>] ... ]

or in case it is an Action:

Link: <Resource URL> + ";action=" + <Term of the Action>;
    rel=<Category identifier of the Action>
\end{verbatim}

\subsubsection{Rendering of OCCI-Attributes}
The X-OCCI-Attribute MUST be used to render the attributes associated
with a OCCI Kind. A simple key-value format is used. The field value
consist of an attribute name followed by an equal sign ("=") and the
attribute value. The attribute value must be quoted if it includes a
separator character, see RFC 2616 (page 16).

\begin{verbatim}
X-OCCI-Attribute: <attribute name>=<attribute value>
\end{verbatim}

Valid attribute names for OCCI Kinds are specified in appropriate
Extension documents.

\subsubsection{Rendering of Location-URLs}
To render an OCCI representation solely in the header, the
X-OCCI-Location header MUST be used to return a list of Kind
URLs. Each header field value correspond to a single URL. Multiple
Kind URLs are returned using multiple X-OCCI-Location headers. See RFC
2616 for information on how to render multiple HTTP headers.

\begin{verbatim}
X-OCCI-Location: <URL>
\end{verbatim}

\subsubsection{Fields}
The following setups show how the Core Model MUST be rendered. Shown
are the fields which MUST be available in a request from the Client or
a response from the Server.

\begin{tabular}{l|l|l|l}
Operation & Required HTTP-Header(s) & Optional HTTP-Header(s) & Notes \\
\hline
Rendering of a Category & Category & N/A & \\
Rendering a list of Categories & Category & N/A & \\
Rendering a list of Kinds & X-OCCI-Location & N/A & \\
Rendering of a Resource & Category & X-OCCI-Attribute, Link & \\
Rendering of an Action & Category, Link & X-OCCI-Attribute & \\
Rendering of a Link & Category, Link & X-OCCI-Attribute & \\
\end{tabular}

\subsection{General HTTP behaviour which is adopted by OCCI}
The following sections deal with some general HTTP feaures which are
adopted by OCCI.

\subsubsection{Security and Authentication}
OCCI does not require that an authentication mechanism be used nor
does it require that client to service communications are secured. It
does recommend that an authentication mechanism be used and that where
appropriate, communications are encrypted using HTTP over TLS. The
authentication mechanisms that CAN be used with OCCI are those that
can be used with HTTP and TLS, for example Basic [REF], Digest [REF]
and OAuth [REF]. If an OCCI service requires authentication the
response to a request that MUST be authenticated must be a HTTP 401
code that indicates the request is authorized. In response to
authenticate the client MUST set a WWW-Authenticate header field and
through this indicate the authentication mechanism.

\subsubsection{Versioning}
Information about what version of OCCI is supported by a provider MUST
be advertised to a client on each response to a client. The version
field in the response MUST include the value OCCI/X.Y, where X is the
major version number and Y is the minor version number of the
implemented OCCI specification. In the case of a HTTP Header
Rendering, the server response should relay versioning information
using the HTTP header name 'Server'.

\begin{verbatim}
HTTP/1.1 202 Accepted
Server: occi-server/1.1 (linux) OCCI/1.0
[...]
\end{verbatim}

Complimenting the service-side behavior of an OCCI implementation, a
client MUST indicate to the OCCI service implementation the version it
expects to interact with. For the clients, the information MUST be
dvertised in the request it issues. In the case of a HTTP Header
Rendering, the client request should relay versioning information in
the 'User-Agent' header. The 'User-Agent' field must include the same
value (OCCI/X.Y) as specified for the Server HTTP header.

\begin{verbatim}
GET <UDN> HTTP/1.1
Host: example.com
User-Agent: occi-client/1.1 (linux) libcurl/7.19.4 OCCI/1.0
[...]
\end{verbatim}

If a server receives a request from a client that supplies a version
number higher than the service supports, the service MUST respond back
to the client with an exception indicating that the requested version
is not implemented. Where a client implements OCCI using a HTTP
transport, the HTTP code 501, not implemented should be used.

\subsubsection{Content-type and Accept headers}
A server MUST react according to the Accept header the client
provides. If non is given - or \textit{*/*} is used - the service MUST
use the Content-type \emph{text/plain}. This is the fall-back
rendering and MUST be implemented. Otherwise the according rendering
MUST be used. Each Rendering SHOULD expose which Accept and
Content-type header fields it can handle. Overall the server MUST
support the \textit{text/occi}, \textit{text/plain} and
\textit{text/uri-list} Content-types.

The server MUST also return the proper Content-type header. If a
client provides information with a Content-Type - the information MUST
be parsed accordingly.

When the Client request a Content-Type will will result in an
incomplete or faulty rendering the Service MUST return the 406 Not
Acceptable HTTP code.

\paragraph{The Content-type text/plain}
While using this rendering with the Content-Type \textit{text/plain}
the information described in section \ref{sec:syntax} MUST be placed
in the HTTP Body.

Each Rendering of an OCCI base type will be placed in the body. Each
entry consists of a name followed by a colon (":") and the field
value. The format of the field value is specified separately for each
of the three header fields, see section \ref{sec:syntax}.

\paragraph{The Content-type text/occi}
While using this rendering with the Content-Type \textit{text/occi}
the information described in section \ref{sec:model_rendering} MUST be
placed in the HTTP Header. The body MUST contain the string 'OK' on
successfull operations.

The HTTP header fields MUST follow the specification in RFC 2616
\cite{rfc2616}. A header field consists of a name followed by a colon
(":") and the field value. The format of the field value is specified
separately for each of the header fields, see section
\ref{sec:syntax}.

\textbf{Limitations: } HTTP header fields MAY appear multiple times in
a HTTP request or response. In order to be OCCI compliant the
specification of multiple message-header fields according to RFC 2616
MUST be fully supported. In essence there are two valid representation
of multiple HTTP header field values. A header field might either
appear several times or as a single header field with a
comma-separated list of field values. Due to implementation issues in
many web frameworks and client libraries it is RECOMMENDED to use the
comma-separated list format for best interoperability.

HTTP header field values which contain separator characters MUST be
properly quoted according to RFC 2616.

Space in the HTTP header section of a HTTP request is a limited
resource. By this, it is noted that many HTTP servers limit the number
of bytes that can be placed in the HTTP Header area. Implementers MUST
be aware of this limitation in their own implementation and take
appropriate measures so that truncation of header data does NOT
occure.

\paragraph{The Content-type text/uri-list}
This Rendering can handle the \textit{text/uri-list} Accept Header. It
will use the Content-type \textit{text/uri-list}.

This rendering cannot render resource instances or \hl{Kind}s or
\hl{Mixin}s directly but just links to them. For concrete rendering of
Kinds and Categories the Content-types \textit{text/occi},
\textit{text/plain} MUST be used. If a request is done with the
\textit{text/uri-list} in the Accept header, while not requesting for
a Listing a Bad Request MUST be returned.

\subsubsection{Return codes}
\label{sec:return_codes}

Not Authorized
Bad Request... on not allow ops, parsing erros
Server Error
Not Found
501 not implemente...on fault version, non support categories

\section{Contributors}

Editors: Andy Edmonds, Thijs Metsch \\
Contributors: Alexander Papaspyrou, Ralf Nyrén, Sam Johnston

\textbf{TBD: Bunch op people missing here - create table\ldots}

\section{Glossary}
\label{sec:glossary}
\begin{tabular}{l|p{12cm}}
Term & Description \\
\hline
\hl{Action} & An OCCI base type. Represent an invocable operation on a \hl{Entity} sub-type instance or collection thereof. \\

\hl{Category} & A type in the OCCI model. The parent type of \hl{Kind}. \\

\hl{Client} & An OCCI client.\\

\hl{Collection} & A set of \hl{Entity} sub-type instances all associated to a particular \hl{Kind} or \hl{Mixin} instance. \\

\hl{Entity} & An OCCI base type. The parent type of \hl{Resource} and \hl{Link}. \\

\hl{Kind} & A type in the OCCI model. A core component of the OCCI classification system. \\

\hl{Link} & An OCCI base type. A \hl{Link} instance associate one \hl{Resource} instance with another. \\

mixin & An instance of the \hl{Mixin} type associated with a {\bf resource
 instance}. The ``mixin'' concept as used by OCCI {\em only} applies to
 instances, never to \hl{Entity} types. \\

\hl{Mixin} & A type in the OCCI model. A core component of the OCCI classification system. \\

\hl{OCCI} & Open Cloud Computing Interface \\

OCCI base type & One of \hl{Entity}, \hl{Resource}, \hl{Link} or \hl{Action}. \\

OGF & Open Grid Forum \\

\hl{Resource} & An OCCI base type. The parent type for all domain-specific resource types. \\

resource instance & An instance of a sub-type of \hl{Entity}. The OCCI
 model defines two sub-types of \hl{Entity}, the \hl{Resource} type and the
 \hl{Link} type. However, the term {\em resource instance} is defined to
 include any instance of a {\em sub-type} of \hl{Resource} or \hl{Link} as
 well. \\

Tag & A \hl{Mixin} instance with no attributes or actions defined. \\

Template & A \hl{Mixin} instance which if associated at resource instantiation
time pre-populate certain attributes. \\

type & One of the types defined by the OCCI model.  The OCCI model types are
 \hl{Category}, \hl{Kind}, \hl{Mixin}, \hl{Action}, \hl{Entity}, \hl{Resource}
 and \hl{Link}. \\

URI & Uniform Resource Identifier \\
URL & Uniform Resource Locator \\
URN & Uniform Resource Name \\
\end{tabular}


\section{Intellectual Property Statement}
The OGF takes no position regarding the validity or scope of any
intellectual property or other rights that might be claimed to pertain
to the implementation or use of the technology described in this
document or the extent to which any license under such rights might or
might not be available; neither does it represent that it has made any
effort to identify any such rights. Copies of claims of rights made
available for publication and any assurances of licenses to be made
available, or the result of an attempt made to obtain a general
license or permission for the use of such proprietary rights by
implementers or users of this specification can be obtained from the
OGF Secretariat.

The OGF invites any interested party to bring to its attention any
copyrights, patents or patent applications, or other proprietary
rights which may cover technology that may be required to practice
this recommendation. Please address the information to the OGF
Executive Director.


\section{Disclaimer}
This document and the information contained herein is provided on an
``As Is'' basis and the OGF disclaims all warranties, express or
implied, including but not limited to any warranty that the use of the
information herein will not infringe any rights or any implied
warranties of merchantability or fitness for a particular purpose.


\section{Full Copyright Notice}
Copyright \copyright ~Open Grid Forum (2009-2014). All Rights Reserved.

This document and translations of it may be copied and furnished to
others, and derivative works that comment on or otherwise explain it
or assist in its implementation may be prepared, copied, published and
distributed, in whole or in part, without restriction of any kind,
provided that the above copyright notice and this paragraph are
included on all such copies and derivative works. However, this
document itself may not be modified in any way, such as by removing
the copyright notice or references to the OGF or other organizations,
except as needed for the purpose of developing Grid Recommendations in
which case the procedures for copyrights defined in the OGF Document
process must be followed, or as required to translate it into
languages other than English.

The limited permissions granted above are perpetual and will not be
revoked by the OGF or its successors or assignees.


\section{References}

\bibliographystyle{IEEEtran}
\bibliography{references}

\end{document}
