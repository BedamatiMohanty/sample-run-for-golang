\documentclass[10pt,a4paper]{article}
\usepackage[utf8]{inputenc}
\usepackage{fullpage}
\usepackage{graphicx}
\usepackage{fancyhdr}
\usepackage{occi}
\setlength{\headheight}{13pt}
\pagestyle{fancy}

% default sans-serif
\renewcommand{\familydefault}{\sfdefault}

% no lines for headers and footers
\renewcommand{\headrulewidth}{0pt}
\renewcommand{\footrulewidth}{0pt}

% header
\fancyhf{}
\lhead{GFD-P-R.185}
\rhead{\today}

% footer
\lfoot{occi-wg@ogf.org}
\rfoot{\thepage}

% paragraphs need some space...
\setlength{\parindent}{0pt}
\setlength{\parskip}{1ex plus 0.5ex minus 0.2ex}

% some space between header and text...
\headsep 13pt

\setcounter{secnumdepth}{4}

\begin{document}

% header on first page is different
\thispagestyle{empty}

GFD-P-R.185 \hfill Thijs Metsch, Platform Computing\\
OCCI-WG \hfill Andy Edmonds, Intel\\
\rightline {October 7, 2010}\\
\rightline {Updated: \today}

\vspace*{0.5in}

\begin{Large}
\textbf{Open Cloud Computing Interface - RESTful HTTP Rendering}
\end{Large}

\vspace*{0.5in}

\underline{Status of this Document}

This document provides information to the community regarding the
specification of the Open Cloud Computing Interface. Distribution is
unlimited.

\underline{Copyright Notice}

Copyright \copyright ~Open Grid Forum (2009-2011). All Rights Reserved.

\underline{Trademarks}

OCCI is a trademark of the Open Grid Forum.

\underline{Abstract}

This document, part of a document series, produced by the OCCI working
group within the Open Grid Forum (OGF), provides a high-level
definition of a Protocol and API. The document is based upon
previously gathered requirements and focuses on the scope of important
capabilities required to support modern service offerings.

% Dealt via errata
% \ralf{In "Operation on Paths in the Name-space" the GET verb is a MUST for 2
% slightly different use cases: "state of name-space hierarchy" and "all
% instances below a path". Since there appears to be no way for a server to
% differentiate between the 2 cases maybe put it all in the same paragraph?}

\newpage
\tableofcontents
\newpage

\section{Introduction}
The Open Cloud Computing Interface (OCCI) is a RESTful Protocol and
API for all kinds of management tasks. OCCI was originally initiated
to create a remote management API for IaaS%
\footnote{Infrastructure as a Service}
model-based services, allowing for the development of interoperable tools for
common tasks including deployment, autonomic scaling and monitoring.
%
It has since evolved into an flexible API with a strong focus on
interoperability while still offering a high degree of extensibility. The
current release of the Open Cloud Computing Interface is suitable to serve many
other models in addition to IaaS, including e.g.~PaaS and SaaS.

In order to be modular and extensible the current OCCI specification is
released as a suite of complimentary documents, which together form the complete
specification.
%
The documents are divided into three categories consisting of the OCCI Core,
the OCCI Renderings and the OCCI Extensions.
%
\begin{itemize}
\item The OCCI Core specification consists of a single document defining the
 OCCI Core Model. The OCCI Core Model can be interacted with {\em
 renderings} (including associated behaviours) and expanded through {\em extensions}.
\item The OCCI Rendering specifications consist of multiple documents each
 describing a particular rendering of the OCCI Core Model. Multiple renderings can
 interact with the same instance of the OCCI Core Model and will automatically support
 any additions to the model which follow the extension rules defined in OCCI
 Core.
\item The OCCI Extension specifications consist of multiple documents each
 describing a particular extension of the OCCI Core Model. The extension documents
 describe additions to the OCCI Core Model defined within the OCCI specification
 suite.
\end{itemize}
%
The current specification consist of three documents.
Future releases of OCCI may include additional rendering and extension
specifications. The documents of the current OCCI specification suite are:

\begin{description}
\item[OCCI Core] describes the formal definition of the the OCCI Core Model
\cite{occi:core}.
\item[OCCI HTTP Rendering] defines how to interact with the OCCI Core Model using the
RESTful OCCI API \cite{occi:http_rendering}. The document defines how the OCCI Core Model can
be communicated and thus serialised using the HTTP protocol.
\item[OCCI Infrastructure] contains the definition of the OCCI Infrastructure
extension for the IaaS domain \cite{occi:infrastructure}. The document defines
additional resource types, their attributes and the actions that can be taken
on each resource type.
\end{description}


\section{Notational Conventions}
All these parts and the information within are mandatory for
implementors (unless otherwise specified). The key words "MUST", "MUST
NOT", "REQUIRED", "SHALL", "SHALL NOT", "SHOULD", "SHOULD NOT",
"RECOMMENDED", "MAY", and "OPTIONAL" in this document are to be
interpreted as described in RFC 2119 \cite{rfc2119}.

\textbf{Andy: we need to state that this document as part of the current document set,
supersedes all previous documents.}


This document uses the Augmented Backus-Naur Form (ABNF) notation of
RFC 2616 \cite{rfc2616}, and explicitly includes the following rules
from it: quoted-string, token, SP (space), LOALPHA, DIGIT.

\subsection{Specification Examples}
All examples in this document use one of the following three HTTP
category definitions. An example URL name-space hierarchy is also
given. Syntax and Semantics are explained in the remaining sections of
the document. These Category examples do not strive to be complete but
to show the features OCCI has:

\begin{verbatim}
Category: compute;
          scheme="http://schemas.ogf.org/occi/infrastructure#";
          class="kind";
          location="http://example.com/compute/"
          [...]
    (This is a compute kind)

Category: networkinterface;
          scheme="http://schemas.ogf.org/occi/infrastructure#";
          class="kind";
          location="http://example.com/link/networkinterface/"
          [...]
    (This is a storage link)

Category: my_stuff;
          scheme="http://example.com/occi/my_stuff#";
          class="mixin";
          location="http://example.com/my_stuff/"
          [...]
    (This is a mixin of user1)
\end{verbatim}
The following URL name-space hierarchy is used in the examples:
\begin{verbatim}
http://example.com/-/
http://example.com/vms/vm3
http://example.com/vms/foo/vm1
http://example.com/vms/bar/vm1
http://example.com/compute/
http://example.com/link/networkinterface/
http://example.com/my_stuff/
\end{verbatim}

The following terms \cite{rfc3986} are used when referring to URI
components:

\begin{verbatim}
 http://example.com:8080/over/there?action=stop#xyz
 \__/   \______________/\_________/ \_________/ \_/
  |           |            |            |        |
scheme     authority       path        query   fragment
\end{verbatim}

\subsection{Specification Examples' Content-type}
All examples in this document use the \emph{text/plain} HTTP
Content-Type for posting information. To retrieve information the HTTP
Accept header \emph{text/plain} is used.

This specification is aligned with RFC 3986 \cite{rfc3986}.

\section{OCCI HTTP Rendering}
The OCCI HTTP Rendering document specifies how the OCCI Core Model
\cite{occi:core}, including extensions thereof, is rendered over the
HTTP protocol \cite{rfc2616}. The document describes the general
behavior for all interaction with an OCCI implementation over HTTP
together with three content types to represent the data being
transferred. The content types specified are:

\begin{itemize}
	\item \textit{text/plain},
	\item \textit{text/occi} and 
	\item \textit{text/uri-list}. 
\end{itemize}

More details are discussed in section \ref{sec:content_type}. Other
data formats such as e.g.~OVF and JSON will be specified in
complimentary documents.

\subsection{Introduction}
The OCCI HTTP Rendering uses many of the features the HTTP and
underlying protocols offer and builds upon the Resource Oriented
Architecture (ROA). ROA's use Representation State Transfer (REST)
\cite{REST_Fielding} to cater for client and service
interactions. Interaction with the system is by inspection and
modification of a set of related resources and their states, be it on
the complete state or a sub-set. Resources MUST be uniquely
identified. HTTP is an ideal protocol to use in ROA systems as it
provides the means to uniquely identify individual resources through
URLs as well as operating upon them with a set of general-purpose
methods known as HTTP verbs. These HTTP verbs map loosely to the
resource related operations of Create (POST), Retrieve (GET), Update
(POST/PUT) and Delete (DELETE).

Each resource instance within an OCCI system MUST have a unique
identifier stored in the \emph{occi.core.id} attribute of the
\hl{Entity} type \cite{occi:core}.
It is RECOMMENDED to use a Uniform Resource Name (URN) as the
identifier stored in \emph{occi.core.id}.

The structure of these identifiers is opaque and the system should not
assume a static, pre-determined scheme for their structure. For
example \emph{occi.core.id} could be
\emph{urn:uuid:de7335a7-07e0-4487-9cbd-ed51be7f2ce4}.

\subsection{Behavior of the HTTP Verbs}
As OCCI adopts a ROA, REST-based architecture and uses HTTP as the
foundation protocol, resource instances are interacted with through
the four main HTTP verbs. OCCI service implementations MUST, at a
minimum, support these verbs as shown in the summary table
\ref{tbl:http_verbs}.

\mytablefloat{\label{tbl:http_verbs}HTTP Verb Behavior Summary (* = Supports filtering mechanisms)}{
\begin{tabular}{p{0.8in} p{0.8in} p{0.8in} p{0.8in} p{0.8in} p{0.8in}}
\toprule
Path 								& GET 										& POST 												& POST (action - Query = ?action=\ldots) 			& PUT 									& DELETE \\
\colrule
resource instance (/vms/foo/vm1) 	& Retrieval of the resource instance's representation 		& Partial update of the resource instance 			& Perform an action 								& Creation/Update of the resource instance, supplying the full representation of the resource instance	& Deletion of the resource instance \\
\colrule
\hl{Kind} collection (/compute/) 	& Retrieve a collection of resource instances* 	& Create a new resource instance of this \hl{Kind}	& Performs actions on a collection of resource instances 	& Not Defined		& Removal of a single, a subset of or all the resource instances from the kind collection \\
\colrule
\hl{Mixin} collection (/my\_stuff/) 	& Retrieve a collection of resource instances* 	& Adds a resource instance to this collection 		& Performs actions on a collection of resource instances 	& Update of the collection supplying the full representation of it. Includes removal and addition of resources. 		& Removal of a single, a subset of or all the resource instances from the Mixin collection \\
\colrule
query interface (/-/) 				& Retrieve capabilities* 					& Add a user-defined \hl{Mixin} 					& Not Defined 										& Not Defined 							& Removal of a user-defined \hl{Mixin} \\
\botrule
\end{tabular}
}

\subsection{A RESTful Rendering of OCCI}
The following sections and paragraphs describe how the OCCI model MUST
be implemented by OCCI implementations. Operations which are not
defined are out of scope for this specification and MAY be implemented. 
This is the minimal set to ensure interoperability.

\subsubsection{Resource instance URL Name-space Hierarchy and Location}

The URL name-space (in this document referred to as name-space) and
the URL hierarchy (in this document referred to as hierarchy) are freely
definable by the Service Provider. The OCCI implementation MUST
implement the location path feature, which is required by the OCCI
Query Interface. Location paths tell the client where all resource
instances of one \hl{Kind} or \hl{Mixin} can be found regardless of
the hierarchy the service provider defines. Location paths are defined
through the HTTP Category rendering and MUST be present for all HTTP
Categories that can be instantiated (i.e. provisioned). The location
paths MUST end with a \emph{'/'}. These paths are discoverable by the
client through the Query interface \ref{sec:query}.

\subsection{Various Operations and their Prerequisites and Behaviors}
\label{sec:behaviours}
For the expected responses to all the behaviors specified in the
sections below, please refer to the HTTP return codes table in section
\ref{sec:return_codes}. This part of the OCCI specification introduces
a filter mechanism. Filtering means that the OCCI implementation MUST
only return resources which match the filtering patterns defined in
the request. This allows clients to be more specific in their request
and limit the amount of data transferred. Clients MAY currently only
filter on \hl{Category} identifiers and resource instance attributes. For
\hl{Category} the client MUST supply a valid Category rendering. For
attributes the client MUST supply and attribute with a specific value.

\subsubsection{Handling the Query Interface}
\label{sec:query}
The query interface MUST be implemented by all OCCI
implementations. It MUST be found at the path \emph{/-/} off the root
of the OCCI implementation. Implementations MAY also adopt RFC 5785
\cite{rfc5785} compliance to advertise this location (see Section
\ref{sec:rfc5785_compliance} for more details).  With the help of the
query interface it is possible for the client to determine the
capabilities of the OCCI implementation he refers to. The following
Query Interface operations are listed below.

\begin{description}
  \item[Retrieval of all Registered \hl{Kind}s, \hl{Action}s and
    \hl{Mixin}s] The HTTP verb GET MUST be used to retrieve all
    \hl{Kind}s, \hl{Action}s and \hl{Mixin}s the service can
    manage. This allows the client to discover the capabilities of the
    OCCI implementation. The result MUST contain all information about
    the \hl{Kind}s, \hl{Action}s and \hl{Mixin}s (including Attributes
    and \hl{Actions} assigned).

\begin{verbatim}
> GET /-/ HTTP/1.1
> [...]

< HTTP/1.1 200 OK
< [...]
< Category: compute;
<           scheme="http://schemas.ogf.org/occi/infrastructure#";
<           class="kind";
<           title="Compute Resource type";
<           rel="http://schemas.ogf.org/occi/core#resource";
<           attributes="occi.compute.cores occi.compute.state{immutable} ...";
<           actions="http://schemas.ogf.org/occi/infrastructure/compute/action#stop ...";
<           location="http://example.com/compute/"
< Category: storage;
<           scheme="http://schemas.ogf.org/occi/infrastructure#";
<           class="kind";
<           title="Storage Resource type";
<           rel="http://schemas.ogf.org/occi/core#resource";
<           attributes="occi.storage.size{required} occi.storage.state{immutable}";
<           actions="...";
<           location="http://example.com/storage/"
< Category: start;
<           scheme="http://schemas.ogf.org/occi/infrastructure/compute/action#";
<           class="action";
<           title="Start Compute Resource";
<           attributes="method"
< Category: stop;
<           scheme="http://schemas.ogf.org/occi/infrastructure/compute/action#";
<           class="action";
<           title="Stop Compute Resource";
<           attributes="method"
< Category: my_stuff;
<           scheme="http://example.com/occi/my_stuff#";
<           class="mixin";
<           location="http://example.com/my_stuff/"
\end{verbatim}

    An OCCI implementation MUST support a filtering mechanism. If one
    or multiple \hl{Categories} are provided in the request the server
    MUST only return the complete rendering of the requested
    \hl{Kind}s or \hl{Mixin}s. The \emph{text/occi} rendering SHOULD
    be used to define the filters in a request.

    The information which needs to be present in the request and
    response are defined as following and MUST be implemented by an
    OCCI implementation.

\begin{verbatim}
request     = filter
  filter    = *( Category CRLF )

response        = categories
  categories    = *( Category CRLF )
\end{verbatim}

  \item[Adding a \hl{Mixin} Definition] To add a
    \hl{Mixin}\footnote{This can be used to 'tag' a set of resource
      instances.} to the service the HTTP POST verb MUST be used. All
    possible information for the \hl{Mixin} MUST be defined. At least
    the HTTP Category term, scheme and location MUST be
    defined. Actions and Attributes are not supported:

\begin{verbatim}
> POST /-/ HTTP/1.1
> [...]
> Category: my_stuff;
>           scheme="http://example.com/occi/my_stuff#";
>           class="mixin";
>           rel="http:/example.com/occi/something_else#mixin";
>           location="/my_stuff/"

< HTTP/1.1 200 OK
< [...]
\end{verbatim}

    The service might reject this request if it does not allow
    user-defined \hl{Mixin}s to be created. Also on name collisions of
    the specified location, scheme, term or rel value the service
    provider MAY reject this operation.

    The information which needs to be present in the request and
    response are defined as following and MUST be implemented by an
    OCCI implementation.

\begin{verbatim}
request     = mixin
  mixin     = 1( Category CRLF )

response    = N/A
\end{verbatim}

  \item[Removing a \hl{Mixin} Definition] A user defined \hl{Mixin}
    MAY be removed (if allowed) by using the HTTP DELETE verb. The
    information about which \hl{Mixin} should be deleted MUST be
    provided in the request:
    
\begin{verbatim}
> DELETE /-/ HTTP/1.1
> [...]
> Category: my_stuff; 
            scheme="http://example.com/occi/my_stuff#"; 
            class="mixin";

< HTTP/1.1 200 OK
< [...]
\end{verbatim}

    The information which needs to be present in the request and
    response are defined as following and MUST be implemented by an
    OCCI implementation.
    
    If a client attempts to remove a provider-defined Mixin category,
    the implementation MUST respond with a HTTP error code of 403,
    Forbidden.
    
    When a user-defined Mixin is removed all associations with
    resource instances must be removed by the implementation.

\begin{verbatim}
request     = mixin
  mixin     = 1( Category CRLF )

response    = N/A
\end{verbatim}

\end{description}

\subsubsection{Operation on Paths in the Name-space}

The following operations are defined when operating on paths in the
name-space hierarchy which are not location paths nor resource
instances. They MUST end with \emph{/} (For example
\emph{http://example.com/vms/foo/}).

\begin{description}

   \item[Retrieving the State of the Name-space Hierarchy] The HTTP
     verb GET MUST be used to retrieve the current state of the
     name-space hierarchy. It MAY include URIs of resource instances,
     Paths in the name-space as well as URLs for \hl{Kind} and
     \hl{Mixin} collections.

   It is RECOMMENDED to use the \emph{text/uri-list} Accept HTTP
   header for this request.

\begin{verbatim}
> GET /vms/ HTTP/1.1
> Accept: text/uri-list
> [...]

< HTTP/1.1 200 OK
< Content-type: text/uri-list
< [...]
<
http://example.com/vms/vm3
http://example.com/vms/foo/
http://example.com/vms/bar/
\end{verbatim}

  \item[Retrieving All resource instances Below a Path] The HTTP verb
    GET MUST be used to retrieve all resource instances. The service
    provider MUST return a listing containing all resource instances
    which are children of the provided URI in the name-space
    hierarchy:

\begin{verbatim}
> GET /vms/foo/ HTTP/1.1
> [...]
 
< HTTP/1.1 200 OK
< [...]
< 
< X-OCCI-Location: http://example.com/vms/foo/vm1
< X-OCCI-Location: http://example.com/vms/bar/vm2
\end{verbatim}

    An OCCI implementations MUST support a filtering mechanism. If a
    Category is provided in the request the server MUST only return
    the resource instances belonging to the provided \hl{Mixin} or
    \hl{Kind}.

    If an OCCI Entity attribute (X-OCCI-Attribute) is provided in the
    request the server MUST only return the resource instances which
    have a matching attribute value. The \emph{text/occi} rendering
    SHOULD be used to define the filters in a request.

    The information which needs to be present in the request and
    response are defined as following and MUST be implemented by an
    OCCI implementation.

\begin{verbatim}
request      = filter
  filter     = *( Category CRLF )
               *( Attribute CRLF )

response                    = resource_representations
  resource_representations  = *( Location CRLF ) 
\end{verbatim}

  \item[Deletion of All resource instances Below a
    Path]\footnote{Note: this is a potentially dangerous operation!}
    The HTTP verb DELETE MUST be used to delete all resource instances
    under a hierarchy:

\begin{verbatim}
> DELETE /vms/foo/ HTTP/1.1 [...]
 
< HTTP/1.1 200 OK
< [...]
\end{verbatim}

\begin{verbatim}
request      = N/A

response     = N/A
\end{verbatim}

\end{description}

\subsubsection{Operations on \hl{Mixin}s or \hl{Kind}s}
\label{sec:mixin_and_kind_operations}
All of the following operations MUST only be be performed on location
paths provided by \hl{Kind}s and \hl{Mixin}s. The path MUST end with
an \emph{/}.

In contrast to the last section it is valid for the operations defined
here to return a 204 HTTP response code with no content. This is used
when wanting to represent an empty collection of \hl{Mixin}s or
\hl{Kind}s.


\begin{description}
  \item[Retrieving All resource instances Belonging to \hl{Mixin} or
    \hl{Kind}] The HTTP verb GET MUST be used to retrieve all resource
    instances. The service provider MUST return a listing containing
    all resource instances which belong to the requested \hl{Mixin} or
    \hl{Kind}:

\begin{verbatim}
> GET /compute/ HTTP/1.1
> [...]
 
< HTTP/1.1 200 OK
< [...]
< 
< X-OCCI-Location: http://example.com/vms/foo/vm1
< X-OCCI-Location: http://example.com/vms/foo/vm2
< X-OCCI-Location: http://example.com/vms/bar/vm1
\end{verbatim}

    An OCCI implementation MUST support a filtering mechanism. If a
    HTTP Category is provided in the request the server MUST only
    return the resource instances belonging to the provided \hl{Kind}
    or \hl{Mixin}. The provided HTTP Category definition SHOULD be
    different from the \hl{Kind} or \hl{Mixin} definition which
    defined the location path used in the request.

    If an OCCI Entity attribute (X-OCCI-Attribute) is provided in the
    request the server MUST only return the resource instances which
    have a matching attribute value. The \emph{text/occi} rendering
    SHOULD be used to define the filters in a request.

    The information which needs to be present in the request and
    response are defined as following and MUST be implemented by an
    OCCI implementation.

\begin{verbatim}
request      = filter
  filter     = *( Category CRLF )
               *( Attribute CRLF )

response                    = resource_representations
  resource_representations  = *( Location CRLF ) 
\end{verbatim}

  \item[Triggering Actions on All Instances of a \hl{Mixin} or
    \hl{Kind}] Actions can be triggered on all resource instances of
    the same \hl{Mixin} or \hl{Kind}. The HTTP POST verb MUST be used
    and the request MUST contain the \hl{Category} defining the
    \hl{Action}.  Additionally the \hl{Action} MUST be defined by the
    \hl{Kind} or \hl{Mixin} which defines the location path which is
    used in the request:

\begin{verbatim}
> POST /compute/?action=stop HTTP/1.1
> [...]
> Category: stop; scheme="[...]"; class="action";
> X-OCCI-Attribute: method="poweroff"

< HTTP/1.1 200 OK
< [...]
\end{verbatim}

    The information which needs to be present in the request and
    response are defined as following and MUST be implemented by an
    OCCI implementation.

\begin{verbatim}
request              = action_definition
  action_definition  = 1( Category CRLF )
                       *( Attribute CRLF )

response     = N/A
\end{verbatim}

  \item[Associate resource instances With a \hl{Mixin}] One or
    multiple resource instances can be associated with a \hl{Mixin}
    using the HTTP POST verb. The URIs which uniquely define the
    resource instance MUST be provided in the request:

\begin{verbatim}
> POST /my_stuff/ HTTP/1.1
> [...]
> X-OCCI-Location: http://example.com/vms/foo/vm1

< HTTP/1.1 200 OK
< [...]
\end{verbatim}

    The information which needs to be present in the request and
    response are defined as following and MUST be implemented by an
    OCCI implementation.

\begin{verbatim}
request      = 1*( resources )
  resources  = *( Location CRLF )

response     = N/A
\end{verbatim}

  \item[Full Update of a \hl{Mixin} Collection] A
    collection consisting of \hl{Mixin}s can be updated
    using the HTTP PUT verb. All URIs which are part of the
    collection MUST be provided along with the request. The URIs which
    uniquely define the resource instances MUST be provided in the
    request:

\begin{verbatim}
> PUT /my_stuff/ HTTP/1.1
> [...]
> X-OCCI-Location: http://example.com/vms/foo/vm1
> X-OCCI-Location: http://example.com/vms/foo/vm2
> X-OCCI-Location: http://example.com/disks/foo/disk1

< HTTP/1.1 200 OK
< [...]
\end{verbatim}

    The information which needs to be present in the request and
    response are defined as following and MUST be implemented by an
    OCCI implementation.

\begin{verbatim}
request      = 1*( resources )
  resources  = *( Location CRLF )

response     = N/A
\end{verbatim}

  \item[Dissociate resource instance(s) From a \hl{Mixin}] One or
    multiple resource instances can be removed from a \hl{Mixin} using
    the HTTP DELETE verb. The URIs which uniquely define the resource
    instance MUST be provided in the request:

\begin{verbatim}
> DELETE /my_stuff/ HTTP/1.1
> [...]
> X-OCCI-Location: http://example.com/vms/foo/vm1
> X-OCCI-Location: http://example.com/vms/foo/vm2
> X-OCCI-Location: http://example.com/disks/foo/disk1

< HTTP/1.1 200 OK
< [...]
\end{verbatim}

    The information which needs to be present in the request and
    response are defined as following and MUST be implemented by an
    OCCI implementation.

\begin{verbatim}
request      = 1*( resources )
  resources  = *( Location CRLF )

response     = N/A
\end{verbatim}

\end{description}

\subsubsection{Operations on resource instances}
\label{sec:ops_on_instances}
The following operations MUST be implemented by the OCCI
implementation for operations on resource instances, i.e.~instances of
either OCCI \hl{Resource}, OCCI \hl{Link} or sub-types thereof. All
resource instances MUST be handled equally (if not stated otherwise)
independent of whether they are instances of the OCCI \hl{Resource}
type or the OCCI \hl{Link} type. A resource instance is uniquely
identified by an URI, for example
\emph{http://example.com/vms/foo/vm1}. \footnote{The path MUST NOT end
  with an '/' - that would mean that a client operates on a path in
  the name-space hierarchy}

\begin{description}
  \item[Creating a resource instance] A request to create a resource
    instance MUST contain one and only one HTTP Category rendering
    which refers to a specific \hl{Kind} instance. This \hl{Kind}
    MUST define the type of the resource instance.  A request 
    MAY also contain one or more HTTP Category
    renderings which refer to different \hl{Mixin} instances. Any such
    \hl{Mixin} instances MUST be applicable (if allowed) to the
    resource instance.  A client MAY be REQUIRED to provide additional
    information in the request based on whether the Kind indicates 
    \emph{required} attributes (see \ref{sec:render_cats}) or not.
    
    A resource instance can be created using two ways - HTTP POST
    or PUT:

\begin{verbatim}
> POST /compute/ HTTP/1.1
> [...]
> 
> Category: compute; scheme="http://schemas.ogf.org/occi/infrastructure#"; class="kind"; 
> X-OCCI-Attribute: occi.compute.cores=2
> X-OCCI-Attribute: occi.compute.hostname="foobar"
> [...]
 
< HTTP/1.1 201 OK
< [...]
< Location: http://example.com/vms/foo/vm1
\end{verbatim}

    The path on which this POST verb is executed MUST be the location
    path of the corresponding \hl{Kind}. The OCCI
    implementation MUST return the URL of the newly created
    resource instance in the HTTP Location header%
    \footnote{The HTTP Location header \cite{rfc2616} must not be confused with
    X-OCCI-Location.}.

    The information which needs to be present in the request and
    response are defined as following and MUST be implemented by an
    OCCI implementation.

\begin{verbatim}
request                    = 1*( resource_representation )
  resource_representation  = 1*( Category CRLF )
                             *( Link CRLF )
                             *( Attribute CRLF )

response                   = 0*1( resource_representation )
  resource_representation  = 1*( Category CRLF )
                             *( Link CRLF )
                             *( Attribute CRLF )
\end{verbatim}

    HTTP PUT can also be used to create a resource instance. In this
    case the client asks the service provider to create a resource
    instance at a unique path in the name-space hierarchy.\footnote{If
      a Service Provider does not want the user to define the path of
      a resource instance it can return a Bad Request return code -
      See section \ref{sec:return_codes}. Service Providers MUST
      ensure that the paths of REST resources stays unique in their
      name-space.}

\begin{verbatim}
> PUT /vms/foo/my_first_virtual_machine HTTP/1.1
> [...]
> 
> Category: compute; scheme="http://schemas.ogf.org/occi/infrastructure#"; "class=kind"; 
> X-OCCI-Attribute: occi.compute.cores=2
> X-OCCI-Attribute: occi.compute.hostname="foobar"
> [...]
 
< HTTP/1.1 200 OK
< [...]
\end{verbatim}

    The information which needs to be present in the request and
    response are defined as following and MUST be implemented by an
    OCCI implementation.

\begin{verbatim}
request                    = 1*( resource_representation )
  resource_representation  = 1*( Category CRLF )
                             *( Attribute CRLF )

response                   = 0*1( resource_representation )
  resource_representation  = 1*( Category CRLF )
                             *( Link CRLF )
                             *( Attribute CRLF )
\end{verbatim}

    The OCCI implementation MAY either return 201 or 200 HTTP return
    codes. If the OCCI implementation returns the 200 HTTP response
    code the full representation (as described in 'Retrieving a
    resource instance' in this section) MUST be
    returned. \textbf{Please note} that the HTTP Location header used
    when the service returns the 201 HTTP response code is defined in
    RFC2616 \cite{rfc2616}.
  
    A created resource instance MUST be added
    to the collection defined by the \hl{Kind}.

  \item[Retrieving a resource instance] The HTTP GET verb is used for
    representation retrieval. It MUST return at least the HTTP
    Category which defines the \hl{Kind} of the resource instance and
    associated attributes. HTTP Links pointing to related resource
    instances, other URI or Actions MUST be included if present. Only
    Actions currently applicable\footnote{For example, it makes little
      sense to render the start action of a resource instance if it is
      already running.} SHOULD be rendered using HTTP Links. The
    Attributes of the resource instance MUST be exposed to the client
    if available.

\begin{verbatim}
> GET /vms/foo/vm1 HTTP/1.1
> [...]
 
< HTTP/1.1 200 OK
< [...]
< Category: compute; scheme="http://schemas.ogf.org/occi/infrastructure#"; class="kind";
< Category: my_stuff; scheme="http://example.com/occi/my_stuff#"; class="mixin";
< X-OCCI-Attribute: occi.compute.cores=2
< X-OCCI-Attribute: occi.compute.hostname="foobar"
< Link: [...]
\end{verbatim}

    The information which needs to be present in the request and
    response are defined as following and MUST be implemented by an
    OCCI implementation.

\begin{verbatim}
request          = N/A

response                 = resource_representation
resource_representation  = 1( Category CRLF )
                           *( Link CRLF )
                           *( Attribute CRLF )
\end{verbatim}

  \item[Partial Update of a resource instance] As this
    specification describes a RESTful service it is RECOMMENDED that
    the client first retrieves the resource instance. Partial updating
    is done using the HTTP POST verb. Only the information (HTTP
    Links, HTTP X-OCCI-Attributes or HTTP categories), which are
    updated MUST be provided along with the
    request. \footnote{Changing the type of the resource instance MUST
      NOT be possible.} If the resource instance updated is derived
    from the OCCI Link type HTTP Links MUST NOT be allowed in the
    request.

\begin{verbatim}
> POST /vms/foo/vm1 HTTP/1.1
> [...]
> 
> X-OCCI-Attribute: occi.compute.memory=4.0
> [...]
 
< HTTP/1.1 200 OK
< [...]
\end{verbatim}

    The information which needs to be present in the request and
    response are defined as following and MUST be implemented by an
    OCCI implementation.

\begin{verbatim}
request                    = 1*( resource_representation )
  resource_representation  = 1*( Category CRLF )
                             *( Link CRLF )
                             *( Attribute CRLF )

response     = N/A

\end{verbatim}
    
    The OCCI implementation MAY either return 201 or 200 HTTP return
    codes. If the OCCI implementation returns the 200 HTTP response
    code the full representation (as described in 'Retrieving a
    resource instance' in this section) MUST be
    returned. \textbf{Please note} that the HTTP Location header used
    when the service returns the 201 HTTP response code is defined in
    RFC2616 \cite{rfc2616}.
    
  \item[Full Update of a resource instance] Before updating a resource
    instance it is RECOMMENDED that the client first retrieves the
    resource instance. Full updating is done using the HTTP PUT verb. The
    client must PUT the full representation, along with modifications,
    of the resource instance that the service supplied in the most
    recent GET. Missing information will result in the deletion of the
    same, if allowed by the implementation.

    HTTP Links MUST NOT be allowed when using the HTTP PUT verb. A request
    containing a HTTP Link MUST result in a 400 Bad Request HTTP response.
    Any OCCI Links previously associated with a OCCI Resource MUST remain
    associated after a successful Full Update operation. This is necessary
    for the PUT operation to be idempotent.

    

\begin{verbatim}
> PUT /vms/foo/vm1 HTTP/1.1
> [...]
> 
> X-OCCI-Attribute: occi.compute.memory=4.0
> [...]
 
< HTTP/1.1 200 OK
< [...]
\end{verbatim}

    The information which needs to be present in the request and
    response are defined as following and MUST be implemented by an
    OCCI implementation.

\begin{verbatim}
request                    = 1*( resource_representation )
  resource_representation  = *( Category CRLF )
                             *( Attribute CRLF )

response     = N/A
\end{verbatim}

    The OCCI implementation MAY either return 201 or 200 HTTP return
    codes. If the OCCI implementation returns the 200 HTTP response
    code the full representation (as described in 'Retrieving a
    resource instance' in this section) MUST be
    returned. \textbf{Please note} that the HTTP Location header used
    when the service returns the 201 HTTP response code is defined in
    RFC2616 \cite{rfc2616}.
    
  \item[Deleting a resource instance] A resource instance can be
    deleted using the HTTP DELETE verb. No other information SHOULD be
    added to the request.\footnote{If the resource instances is an
      OCCI \hl{Link} type the source and target \hl{Resource}s MUST be
      updated accordingly.}

\begin{verbatim}
> DELETE /vms/foo/vm1 HTTP/1.1
> [...]

< HTTP/1.1 200 OK
< [...]
\end{verbatim}

    The information which needs to be present in the request and
    response are defined as following and MUST be implemented by an
    OCCI implementation.

\begin{verbatim}
request      = N/A

response     = N/A
\end{verbatim}

  \item[Triggering an Action on a resource instance] To trigger an
    Action on a resource instance the request MUST containing the HTTP
    Category defining the \hl{Action}. It MAY include HTTP
    X-OCCI-Attributes which are the parameters of the Action. Actions
    are triggered using the HTTP POST verb and by adding a query to
    the URI. This query exposes the term of the \hl{Action}. If an
    Action is not available a Bad Request should be returned.

\begin{verbatim}
> POST /vms/foo/vm1?action=stop HTTP/1.1
> [...]
> Category: stop; scheme="[...]"; class="action";
> X-OCCI-Attribute: method="poweroff"

< HTTP/1.1 200 OK
< [...]
\end{verbatim}

    The information which needs to be present in the request and
    response are defined as following and MUST be implemented by an
    OCCI implementation.

\begin{verbatim}
request              = action_definition
  action_definition  = 1( Category CRLF )
                       *( Attribute CRLF )

response     = N/A
\end{verbatim}

\end{description}

\subsubsection{Handling \hl{Link} instances}
Some exceptions on the creation and handling of \hl{Link} resource instances%
\footnote{A \hl{Link} resource instance is an instance of the OCCI \hl{Link}
type or a sub-type thereof.}
are described in this section. They MUST be implemented by an OCCI
implementation.

\begin{description}
  \item[Inline Creation of a \hl{Link} Instance] When creating an
    instance of the OCCI \hl{Resource} type and \hl{Link}s are defined
    in the request, those \hl{Link}s MUST be created implicitly. This
    results in the creation of multiple REST resources. However, only
    the Location of the REST resource which represents the requested
    \hl{Kind} MUST be returned. The URIs of the \hl{Link}s can be
    discovered by retrieving a rendering of the resource
    instance. Attributes for the \hl{Link} resource instance MUST be
    specified in the \hl{Link} rendering during the creation of the
    resource instance. It is NOT recommended to supply the 'self'
    parameter to inline Link representation.

\begin{verbatim}
> POST /compute/ HTTP/1.1
> [...]
> 
> Category: compute; 
            scheme="http://schemas.ogf.org/occi/infrastructure#"; 
            class="kind"; 
> Link: </network/123>;
        rel="http://schemas.ogf.org/occi/infrastructure#network";
        category="http://schemas.ogf.org/occi/infrastructure#networkinterface";
        occi.networkinterface.interface="eth0";
        occi.networkinterface.mac="00:11:22:33:44:55";
> X-OCCI-Attribute: occi.compute.cores=2
> X-OCCI-Attribute: occi.compute.hostname="foobar"
> [...]
 
< HTTP/1.1 200 OK
< [...]
< Location: http://example.com/vms/foo/vm1
\end{verbatim}

  \item[Retrieving \hl{Resource} instances With Defined \hl{Link}s]
    When an resource instance of the OCCI \hl{Resource} type is
    rendered it MUST expose all its owned \hl{Link}s. Since \hl{Link}s
    are directed only those originating outward SHOULD be listed.

\begin{verbatim}
> GET /vms/foo/vm1 HTTP/1.1
> [...]
 
< HTTP/1.1 200 OK
< [...]
< Category: compute; scheme="http://schemas.ogf.org/occi/infrastructure#"; class="kind";
< Category: my_stuff; scheme="http://example.com/occi/my_stuff#"; class="mixin";
< X-OCCI-Attribute: occi.compute.cores=2
< X-OCCI-Attribute: occi.compute.hostname="foobar"
< Link: </network/123>;
        rel="http://schemas.ogf.org/occi/infrastructure#network";
        self="/link/networkinterface/456";
        category="http://schemas.ogf.org/occi/infrastructure#networkinterface";
        occi.networkinterface.interface="eth0";
        occi.networkinterface.mac="00:11:22:33:44:55";
        occi.networkinterface.state="active";
\end{verbatim}

  \item[Creation of \hl{Link} resource instances] To directly create a
    \hl{Link} between two existing resource instances the \hl{Kind} as
    well as a occi.core.source and occi.core.target attribute MUST be
    provided during creation of the \hl{Link} instance\footnote{See
      section \ref{sec:ops_on_instances}}.

\begin{verbatim}
> POST /link/networkinterface/ HTTP/1.1
> [...]
> 
> Category: networkinterface; 
            scheme="http://schemas.ogf.org/occi/infrastructure#"; 
            class="kind"; 
> X-OCCI-Attribute: occi.core.source="http://example.com/vms/foo/vm1"
> X-OCCI-Attribute: occi.core.target="http://example.com/network/123"
> [...]
 
< HTTP/1.1 200 OK
< [...]
< Location: http://example.com/link/networkinterface/456
\end{verbatim}

    The information which needs to be present in the request and
    response are defined as following and MUST be implemented by an
    OCCI implementation.

\begin{verbatim}
request                    = resource_representation
resource_representation    = *( Category CRLF )
                             *( Attribute CRLF )

response                   = 0*1( resource_representation )
\end{verbatim}

  \item[Retrieval of \hl{Link} resource instances] Retrieval of a
    \hl{Link} is the same to the retrieval of any other resource
    instance. Please review section \ref{sec:ops_on_instances} for
    more details.

\begin{verbatim}
> GET /link/networkinterface/456 HTTP/1.1
> [...]

< HTTP/1.1 200 OK
< [...]
< Category: networkinterface;
      scheme="http://schemas.ogf.org/occi/infrastructure#";
      class="kind";
< X-OCCI-Attribute: occi.networkinterface.interface="eth0";
< X-OCCI-Attribute: occi.networkinterface.mac="00:11:22:33:44:55";
< X-OCCI-Attribute: occi.networkinterface.state="active";
< X-OCCI-Attribute: occi.core.source="/vms/foo/vm1"
< X-OCCI-Attribute: occi.core.target="/network/123"
\end{verbatim}

    The information which needs to be present in the request and
    response are defined as following and MUST be implemented by an
    OCCI implementation.

\begin{verbatim}
request                    = N/A

response                   =  1*( resource_representation )
  resource_representation  = *( Category CRLF )
                             *( Attribute CRLF )
\end{verbatim}

\end{description}

\subsection{Syntax and Semantics of the Rendering}
\label{sec:syntax}
All data transferred using the \textit{text/occi} and
\textit{text/plain} content types is structured text. These rendering
structures are compliant with and follow the rules of HTTP headers
\cite{rfc2616}.  Four specific rendering structures are only ever
used:

\begin{itemize}
  \item Category
  \item Link
  \item X-OCCI-Attribute
  \item X-OCCI-Location
\end{itemize}

The \textit{text/occi} content type renders these rendering structures
as HTTP headers in the header portion of a HTTP request or
response. The \textit{text/plain} content type renders the same
rendering structures, with identical syntax, in the body of the HTTP
request/response. See section \ref{sec:content_type} for more
information on the use of different content types.

Multiple field values per rendering structure MUST be supported as
defined by RFC~2616 \cite{rfc2616}. This applies to both the
\textit{text/occi} and \textit{text/plain} content types. RFC 2616
defines two different methods to render multiple header field values,
either a comma-separated list or multiple header lines. The following
two rendering examples are identical and both formats MUST be
supported by both OCCI client and server to be compliant.

Comma-separated rendering of multiple HTTP header field values:

\begin{verbatim}
X-OCCI-Attribute: occi.compute.memory=2.0, occi.compute.speed=2.33
X-OCCI-Location: /compute/123, /compute/456
\end{verbatim}

Separate header lines for each HTTP header field value:
\begin{verbatim}
X-OCCI-Attribute: occi.compute.memory=2.0
X-OCCI-Attribute: occi.compute.speed=2.33
X-OCCI-Location: /compute/123
X-OCCI-Location: /compute/456
\end{verbatim}

\subsubsection{Rendering of the OCCI Category, Kind and Mixin Types}
\label{sec:render_cats}
Instances of the \hl{Category}, \hl{Kind} and \hl{Mixin} types
\cite{occi:core} MUST be rendered using the Category header as defined
by the Web Categories specification%
\footnote{http://tools.ietf.org/html/draft-johnston-http-category-header-01}.

The following syntax applies:

\begin{verbatim}
Category             = "Category" ":" #category-value
  category-value     = term
                      ";" "scheme" "=" <"> scheme <">
                      ";" "class" "=" ( class | <"> class <"> )
                      [ ";" "title" "=" quoted-string ]
                      [ ";" "rel" "=" <"> type-identifier <"> ]
                      [ ";" "location" "=" <"> URI <"> ]
                      [ ";" "attributes" "=" <"> attribute-list <"> ]
                      [ ";" "actions" "=" <"> action-list <"> ]
  term               = LOALPHA *( LOALPHA | DIGIT | "-" | "_" )
  scheme             = URI
  type-identifier    = scheme term
  class              = "action" | "mixin" | "kind"
  attribute-list     = attribute-def
                     | attribute-def *( 1*SP attribute-def)
  attribute-def      = attribute-name
                     | attribute-name
                       "{" attribute-property *( 1*SP attribute-property ) "}"
  attribute-property = "immutable" | "required"
  attribute-name     = attr-component *( "." attr-component )
  attr-component     = LOALPHA *( LOALPHA | DIGIT | "-" | "_" )
  action-list        = action
                     | action *( 1*SP action )
  action             = type-identifier
\end{verbatim}

The {\tt attributes} parameter is used in the OCCI Query Interface,
see section~\ref{sec:query}, to inform a client which attributes a
particular \hl{Entity} sub-type supports. This information is used by
the client to determine which attributes to include in an initial POST
request to create a Resource.

In addition to the existence of an attribute the Query Interface also
informs a client whether an attribute is required or immutable, see
the ``attribute-property'' in the syntax description above. The logic
of attribute properties is as follows:

\begin{itemize} 
  \item If no attribute properties are defined the attribute is
    mutable and non-required, i.e. multiplicity is 0..x.
  \item If the ``{\tt immutable}'' property is set the attribute is
    immutable (not modifiable by the client).
  \item If the ``{\tt required}'' property is set the attribute has a
    multiplicity of 1..x, i.e.~it MUST be specified by the client.
\end{itemize}

The following example illustrates a rendering of the \hl{Kind}
instance assigned to the \hl{Storage} type \cite{occi:infrastructure}:

\begin{verbatim}
Category: storage;
    scheme="http://schemas.ogf.org/occi/infrastructure#";
    class="kind";
    title="Storage Resource";
    rel="http://schemas.ogf.org/occi/core#resource";
    location="/storage/";
    attributes="occi.storage.size{required} occi.storage.state{immutable}";
    actions="http://schemas.ogf.org/occi/infrastructure/storage/action#resize ...";
\end{verbatim}

\subsubsection{Rendering of OCCI Link Instance References}
The rendering of a \hl{resource} instance \cite{occi:core} MUST
represent any associated \hl{Link} instances using the HTTP Link
header specified in the Web Linking RFC 5988 \cite{rfc5988}.  For
example, rendering of a \hl{Compute} instance linked to a \hl{Storage}
instance MUST include a Link header displaying the OCCI \hl{Link}
instance of the relation.

The following syntax MUST be used to represent OCCI \hl{Link} type
instance references:

\begin{verbatim}
Link               = "Link" ":" #link-value
  link-value       = "<" URI-Reference ">"
                    ";" "rel" "=" <"> resource-type <">
                    [ ";" "self" "=" <"> link-instance <"> ]
                    [ ";" "category" "=" link-type
                      *( ";" link-attribute ) ]
  term             = LOALPHA *( LOALPHA | DIGIT | "-" | "_" )
  scheme           = URI
  type-identifier  = scheme term
  resource-type    = type-identifier *( 1*SP type-identifier )
  link-type        = type-identifier *( 1*SP type-identifier )
  link-instance    = URI-reference
  link-attribute   = attribute-name "=" ( token | quoted-string )
  attribute-name   = attr-component *( "." attr-component )
  attr-component   = LOALPHA *( LOALPHA | DIGIT | "-" | "_" )
\end{verbatim}

The following example illustrates the rendering of a
\hl{NetworkInterface} \cite{occi:infrastructure} instance linking to a
\hl{Network} resource instance:

\begin{verbatim}
Link: </network/123>;
    rel="http://schemas.ogf.org/occi/infrastructure#network";
    self="/link/networkinterface/456";
    category="http://schemas.ogf.org/occi/infrastructure#networkinterface";
    occi.networkinterface.interface="eth0";
    occi.networkinterface.mac="00:11:22:33:44:55";
    occi.networkinterface.state="active";
\end{verbatim}

\subsubsection{Rendering of References to OCCI Action Instances}
The rendering of a \hl{Resource} instance \cite{occi:core} MUST
represent any associated \hl{Action} instances using the HTTP Link
header specified in the Web Linking RFC 5988 \cite{rfc5988}.  For
example, rendering of a \hl{Compute} instance MUST include a Link
header displaying any \hl{Action}s currently applicable to the
resource instance.

The following syntax MUST be used to represent OCCI \hl{Action}
instance references:

\begin{verbatim}
Link               = "Link" ":" #link-value
  link-value       = "<" action-uri ">"
                    ";" "rel" "=" <"> action-type <">
  term             =  LOALPHA *( LOALPHA | DIGIT | "-" | "_" )
  scheme           = relativeURI
  type-identifier  = scheme term
  action-type      = type-identifier
  action-uri       = URI "?" "action=" term
\end{verbatim}

The following example illustrates the rendering of a reference to the
``start'' \hl{Action} defined for the \hl{Compute} type
\cite{occi:infrastructure}. Such a reference would be present in the
rendering of a \hl{Compute} instance.

\begin{verbatim}
Link: </compute/123?action=start>;
    rel="http://schemas.ogf.org/occi/infrastructure/compute/action#start"
\end{verbatim}

\subsubsection{Rendering of OCCI Entity Attributes}
Attributes defined for OCCI \hl{Entity} sub-types \cite{occi:core},
i.e.~\hl{Resource} and \hl{Link}, MUST be rendered using the
X-OCCI-Attribute HTTP header. For example the rendering of a
\hl{Compute} instance MUST render the associated attributes, such as
e.g. \texttt{occi.compute.memory}, using X-OCCI-Attribute headers.

The X-OCCI-Attribute header uses a simple key-value format where each
HTTP header field value represent a single attribute. The field value
consist of an attribute name followed by the equal sign (``='') and an
attribute value.

The following syntax MUST be used to represent OCCI \hl{Entity}
attributes:

\begin{verbatim}
Attribute          = "X-OCCI-Attribute" ":" #attribute-repr
  attribute-repr   = attribute-name "=" ( string | number | bool | enum_val )
  attribute-name   = attr-component *( "." attr-component )
  attr-component   = LOALPHA *( LOALPHA | DIGIT | "-" | "_" )	
  string           = quoted-string
  number           = (int | float)
  int              = *DIGIT
  float            = *DIGIT "." *DIGIT
  bool             = ("true" | "false")
  enum_val         = string
\end{verbatim}

Attribute names for the infrastructure types are defined in the OCCI
Infrastructure document \cite{occi:infrastructure}.  The rules for
defining new attribute names can be found in the ``Extensibility''
section of the OCCI Core document \cite{occi:core}.

The following example illustrates a rendering of the attributes
defined by \hl{Compute} type \cite{occi:infrastructure}:

\begin{verbatim}
X-OCCI-Attribute: occi.compute.architecture="x86_64"
X-OCCI-Attribute: occi.compute.cores=2
X-OCCI-Attribute: occi.compute.hostname="testserver"
X-OCCI-Attribute: occi.compute.speed=2.66
X-OCCI-Attribute: occi.compute.memory=3.0
X-OCCI-Attribute: occi.compute.state="active"
\end{verbatim}

\subsubsection{Rendering of Location-URIs}
In order to render an OCCI representation solely in the HTTP header,
i.e.~using the \textit{text/occi} content type, the X-OCCI-Location 
\footnote{This was introduced as the HTTP Location Header can have
  only one value where as this can have multiple.}  HTTP header MUST
be used to return a list of resource instance URIs. Each header field
value correspond to a single URI. Multiple resource instance URIs are
returned using multiple X-OCCI-Location headers.

\begin{verbatim}
Location      = "X-OCCI-Location" ":" location-value
  location-value  = URI-reference
\end{verbatim}

The following example illustrates the rendering of a list of
\hl{Compute} resource instances:
\begin{verbatim}
X-OCCI-Location: http://example.com/compute/123
X-OCCI-Location: http://example.com/compute/456
X-OCCI-Location: http://example.com/compute/789
\end{verbatim}

\subsection{General HTTP Behaviors Adopted by OCCI}
The following sections deal with some general HTTP features which are
adopted by OCCI.

\subsubsection{Security and Authentication}
OCCI does not require that an authentication mechanism be used nor
does it require that client to service communications are secured. It
does RECOMMEND that an authentication mechanism be used and that where
appropriate, communications are encrypted using HTTP over TLS. The
authentication mechanisms that MAY be used with OCCI are those that
can be used with HTTP and TLS. For further discussion see Section
\ref{sec:sec_consid}.

\subsubsection{Additional Headers (Caching Headers)}
The responses from an OCCI implementation MAY include additional
headers like those for caching purposes like E-Tags.

\subsubsection{Asynchronous Operations}
\label{sec:async_ops}
OCCI implementations MAY implement a way to deal with asynchronous
calls. Upon long-running operations the OCCI implementation MAY return
a temporary resource (e.g. a task resource) using the HTTP Location
header and a corresponding HTTP 202 return code. Clients can query
that resource until the operation finishes. Upon completion of the
operation this temporary result will redirect to the resulting REST
resource using the HTTP Location header and return the HTTP 301 return
code signalling the completion.

\subsubsection{Batch operations}
Batch operations, like the ones in described in section
\ref{sec:mixin_and_kind_operations}, are atomic. All parts of the
request MUST be processed - no partial execution is allowed.

\subsubsection{Versioning}
Information about what version of OCCI is supported by a OCCI
implementation MUST be advertised to a client on each response to a
client. The version field in the response MUST include the value
OCCI/X.Y, where X is the major version number and Y is the minor
version number of the implemented OCCI specification. The server
response MUST relay versioning information using the HTTP Server
header.

\begin{verbatim}
HTTP/1.1 200 OK
Server: occi-server/1.1 (linux) OCCI/1.1
[...]
\end{verbatim}

Complimenting the service-side behavior of an OCCI implementation, a
client SHOULD indicate to the OCCI implementation the version it
expects to interact with. For the clients, the information SHOULD be
advertised in all requests it issues. A client request SHOULD relay
versioning information in the 'User-Agent' header. The 'User-Agent'
field MUST include the same value (OCCI/X.Y) as supported by the HTTP
Server header.

\begin{verbatim}
GET <Path> HTTP/1.1
Host: example.com
User-Agent: occi-client/1.1 (linux) libcurl/7.19.4 OCCI/1.1
[...]
\end{verbatim}

If an OCCI implementation receives a request from a client that
supplies a version number higher than the service supports, the
service MUST respond back to the client with an exception indicating
that the requested version is not implemented. Where a client
implements OCCI using a HTTP transport, the HTTP code 501, not
implemented, MUST be used.

OCCI implementations which implement this version of this Document
MUST use the version string \emph{OCCI/1.1}. Versioning of extensions
is out of scope for this version of the document.

\subsubsection{Content-type and Accept headers}
\label{sec:content_type}
A server MUST react according to the Accept header the client
provides. A client SHOULD provide the Accept header in a request. If
none is given - or \textit{*/*} is used - the service MUST use the
Content-type \emph{text/plain}. This is the default and fall-back
rendering and MUST be implemented. Otherwise the according rendering
MUST be used. Each Rendering SHOULD expose which Accept and
Content-type header fields it can handle. Overall the service MUST
support \textit{text/occi}, \textit{text/plain} and
\textit{text/uri-list}.

The server MUST return the proper Content-type header. A client MUST
provide the proper Content-Type when data is send to the OCCI
implementation - the information MUST be parsed accordingly.

When the Client requests a Content-Type that will result in an
incomplete or faulty rendering the Service MUST return the 406 'Not
acceptable' HTTP code.

The following examples demonstrate the behavior of an HTTP GET
operations on the resource instance \emph{} using two different HTTP
Accept headers:

\begin{verbatim}
> GET /vms/foo/vm1 HTTP/1.1
> Accept: text/plain
> [...]

< HTTP/1.1 200 OK
< [...]
<
< Category: compute; scheme="http://schemas.ogf.org/occi/infrastructure#"; class="kind";
< Category: my_stuff; scheme="http://example.com/occi/my_stuff#"; class="mixin"; 
< X-OCCI-Attribute: occi.compute.cores=2
< X-OCCI-Attribute: occi.compute.hostname="foobar"
< Link: [...]
\end{verbatim}

And with \emph{text/occi} as HTTP Accept header:

\begin{verbatim}
> GET /vms/foo/vm1 HTTP/1.1
> Accept: text/occi
> [...]

< HTTP/1.1 200 OK
< Category: compute; scheme="http://schemas.ogf.org/occi/infrastructure#"; class="kind";,
            my_stuff; scheme="http://example.com/occi/my_stuff#"; class="mixin";
< X-OCCI-Attribute: occi.compute.cores=2, occi.compute.hostname=foobar
< Link: [...]
< [...]
<
< OK
\end{verbatim}

\paragraph{The Content-type text/plain}
While using this rendering with the Content-Type \textit{text/plain}
the information described in section \ref{sec:syntax} MUST be placed
in the HTTP Body.

Each rendering of an OCCI base type will be placed in the body. Each
entry consists of a name followed by a colon (":") and the field
value. The format of the field value is specified separately for each
of the three header fields, see section \ref{sec:syntax}.

\paragraph{The Content-type text/occi}
While using this rendering with the Content-Type \textit{text/occi}
the information described in section \ref{sec:syntax} MUST be placed
in the HTTP Header. The body MUST contain the string 'OK' on
successful operations.

The HTTP header fields MUST follow the specification in RFC 2616
\cite{rfc2616}. A header field consists of a name followed by a colon
(":") and the field value. The format of the field value is specified
separately for each of the header fields, see section
\ref{sec:syntax}.

\textbf{Limitations: } HTTP header fields MAY appear multiple times in
a HTTP request or response. In order to be OCCI compliant, the
specification of multiple message-header fields according to RFC 2616
MUST be fully supported. In essence there are two valid representation
of multiple HTTP header field values. A header field might either
appear several times or as a single header field with a
comma-separated list of field values. Due to implementation issues in
many web frameworks and client libraries it is RECOMMENDED to use the
comma-separated list format for best interoperability.

HTTP header field values which contain separator characters MUST be
properly quoted according to RFC 2616.

Space in the HTTP header section of a HTTP request is a limited
resource. By this, it is noted that many HTTP servers limit the number
of bytes that can be placed in the HTTP Header area. Implementers MUST
be aware of this limitation in their own implementation and take
appropriate measures so that truncation of header data does NOT occur.

\paragraph{The Content-type text/uri-list}
This Rendering can handle the \textit{text/uri-list} Accept Header. It
will use the Content-type \textit{text/uri-list}.

This rendering cannot render resource instances or \hl{Kind}s or
\hl{Mixin}s directly but just links to them. For concrete rendering of
Kinds and Categories the Content-types \textit{text/occi},
\textit{text/plain} MUST be used. If a request is done with the
\textit{text/uri-list} in the Accept header, while not requesting for
a Listing a Bad Request MUST be returned. Otherwise a list of
resources MUST be rendered in \emph{text/uri-list} format, which can
be used for listing resource in collections or the name-space of the
OCCI implementation.

\subsubsection{RFC5785 Compliance}
\label{sec:rfc5785_compliance}
Should implementations wish to advertise the Query Interface using the
.well-known mechanism defined by "Defining Well-Known Uniform Resource
Identifiers" \cite{rfc5785} then they MUST use the following path
served from the authority:

/.well-known/org/ogf/occi/-/

The functionality accessible at this location MUST exactly mirror that
as defined in section \ref{sec:query} on the Query Interface.

\subsubsection{Return Codes}
\label{sec:return_codes}
At any point the service provider MAY return any of the following HTTP
Return Codes in table. These codes are for behaviors in section
\ref{sec:behaviours} where appropriate \ref{tbl:http_codes}

\mytablefloat{\label{tbl:http_codes}HTTP Return Codes}{
\begin{tabular}{l|l|p{3in}}
\toprule
Code & Description & Notes \\
\colrule
200 & OK & Indicates that the request was successful. The response MUST contain the created resource instance's representation.\\
201 & OK & Indicates that the request was successful. The response MUST contain a HTTP Location header to the newly created resource instance.\\
202 & Accepted & Used for asynchronous non-blocking calls. See section \ref{sec:async_ops}.\\
204 & OK, but no content returned. & This is used to indicate that a collection is empty.\\
400 & Bad Request & Used to signal parsing errors or missing information (e.g. an attribute that is required is not supplied in the request). This applies also to filters.  \\
401 & Unauthorized & The client does not have the required permissions/credentials.\\
403 & Forbidden & Used to signal that a particular Mixin cannot be applied to a resource instance of a particular Kind. Used to signal that an attempt was made to modify an attribute that was marked as immutable.\\
404 & Not Found & Used to signal that the request had information (e.g. a kind, mixin, action, attribute, location) that was unknown to the service and so not found.\\
405 & Method Not Allowed & The service does not allow the client to issue the HTTP method against the requested path/location\\
406 & Not Acceptable & See section \ref{sec:content_type} \\
409 & Conflict & A request contains content (e.g. mixin, kind, action) that results in an internal service, non-unique result (e.g. two types of start actions are found for Compute). The client MUST resolve the conflict by re-trying with specific Category information in the request.\\
410 & Gone & A client attempts to retrieve a resource instance that no longer exists (i.e. it was deleted).\\
% 415 & Unsupported Media Type & See section \ref{sec:content_type}\\
500 & Internal Server Error & The state before the request should be maintained in such an error condition. The implementation MUST roll-back any partial changes made during the erroneous execution.\\
501 & Not Implemented & If an implementation chooses not to implement a particular OCCI feature, it MUST signal the lack of that feature with this code. This implicitly points to a non-compliant OCCI implementation.\\
503 & Service Unavailable & If the OCCI service is taken down for maintenance, this error code should be reported from the root of the name-space the provider uses.\\
\botrule
\end{tabular}}

\subsection{More Complete Examples}
Since most examples are not complete due to space limitations this
section will give some more complete examples.

\subsubsection{Creating a Compute resource instance}
\begin{verbatim}
> POST /compute/ HTTP/1.1
> User-Agent: curl/7.21.0 (x86_64-pc-linux-gnu) libcurl/7.21.0 OpenSSL/0.9.8o zlib/1.2.3.4 libidn/1.18
> Host: localhost:8080
> Accept: */*
> Content-Type: text/occi
> Category: compute; scheme="http://schemas.ogf.org/occi/infrastructure#"; class="kind";
>
< HTTP/1.1 200 OK
< Content-Length: 2
< Content-Type: text/plain; charset=UTF-8
< Location: http://example.com/users/foo/compute/b9ff813e-fee5-4a9d-b839-673f39746096
< Server: example-occi OCCI/1.1
<
< OK
\end{verbatim}

\subsubsection{Retrieving a Compute resource instance}
\begin{verbatim}
> GET /users/foo/compute/b9ff813e-fee5-4a9d-b839-673f39746096 HTTP/1.1
> User-Agent: curl/7.21.0 (x86_64-pc-linux-gnu) libcurl/7.21.0 OpenSSL/0.9.8o zlib/1.2.3.4 libidn/1.18
> Host: localhost:8080
> Accept: */*
>
< HTTP/1.1 200 OK
< Content-Length: 642
< Etag: "ef485dc7066745cb0fe1e31ecdd4895c356b5bd5"
< Content-Type: text/plain
< Server: example-occi OCCI/1.1
<
< Category: compute;
<     scheme="http://schemas.ogf.org/occi/infrastructure#"
<     class="kind";
< Link: </users/foo/compute/b9ff813e-fee5-4a9d-b839-673f39746096?action=start>;
<     rel="http://schemas.ogf.org/occi/infrastructure/compute/action#start"
< X-OCCI-Attribute: occi.core.id="urn:uuid:b9ff813e-fee5-4a9d-b839-673f39746096"
< X-OCCI-Attribute: occi.core.title="My Dummy VM"
< X-OCCI-Attribute: occi.compute.architecture="x86"
< X-OCCI-Attribute: occi.compute.state="inactive"
< X-OCCI-Attribute: occi.compute.speed=1.33
< X-OCCI-Attribute: occi.compute.memory=2.0
< X-OCCI-Attribute: occi.compute.cores=2
< X-OCCI-Attribute: occi.compute.hostname="dummy"
\end{verbatim}

\section{OCCI Compliance Tools}
For the ease of generating compliant parsers an
ANTLR\footnote{http://www.antlr.org} grammar is available
from\footnote{http://github.com/dizz/occi-grammar} where it is updated
and maintained. It is based on the ABNF grammars quoted through out
the document. It can parse the OCCI text renderings of Category, Link,
X-OCCI-Attribute and X-OCCI-Location. It is recommended that
implementers use this grammar as a means to generate OCCI compliant
parsers.

To verify an implementations run-time behavior, the OCCI Compliance
Testing Tool is provided
\footnote{http://forge.ogf.org/sf/scm/do/listRepositories/projects.occi-wg/scm}.
This tool uses the OCCI ANTLR generated parser along with various HTTP
libraries to ensure that a OCCI implementation reacts and responds
correctly to valid OCCI client requests. It is recommended that
implementers use this tool to aid them in validating compliance with
OCCI.

\section{Security Considerations}
\label{sec:sec_consid}
The OCCI HTTP rendering assumes HTTP or HTTP-related mechanisms for
security. As such, implementations SHOULD support
TLS \footnote{http://datatracker.ietf.org/wg/tls/} for transport layer
security.

Authentication SHOULD be realized by HTTP authentication mechanisms,
namely HTTP Basic or Digest Auth \cite{rfc2617}, with the former as
default. Additional profiles MAY specify other methods and should
ensure that the selected authentication scheme can be renderable over
the HTTP or HTTP-related protocols.

Authorization is not enforced on the protocol level, but SHOULD be
performed by the implementation. For the authorization decision, the
authentication information as provided by the mechanisms described
above MUST be used.

Protection against potential Denial-of-Service scenarios are out of
scope of this document; the OCCI HTTP Rendering specifications assumes
cooperative clients that SHOULD use selection and filtering as
provided by the Category mechanism wherever possible. Additional
profiles to this document, however, MAY specifically address such
scenarios; in that case, best practices from the HTTP ecosystem and
appropriate mechanisms as part of the HTTP protocol specification
SHOULD be preferred.

As long as specific extensions of the OCCI Core and Model
specification do not impose additional security requirements than the
OCCI Core and Model specification itself, the security considerations
documented above apply to all (existing and future)
extensions. Otherwise, an additional profile to this specification
MUST be provided; this profile MUST express all additional security
considerations using HTTP mechanisms.

\section{Glossary}
\label{sec:glossary}
\begin{tabular}{l|p{12cm}}
Term & Description \\
\hline
\hl{Action} & An OCCI base type. Represent an invocable operation on a \hl{Entity} sub-type instance or collection thereof. \\

\hl{Category} & A type in the OCCI model. The parent type of \hl{Kind}. \\

\hl{Client} & An OCCI client.\\

\hl{Collection} & A set of \hl{Entity} sub-type instances all associated to a particular \hl{Kind} or \hl{Mixin} instance. \\

\hl{Entity} & An OCCI base type. The parent type of \hl{Resource} and \hl{Link}. \\

\hl{Kind} & A type in the OCCI model. A core component of the OCCI classification system. \\

\hl{Link} & An OCCI base type. A \hl{Link} instance associate one \hl{Resource} instance with another. \\

mixin & An instance of the \hl{Mixin} type associated with a {\bf resource
 instance}. The ``mixin'' concept as used by OCCI {\em only} applies to
 instances, never to \hl{Entity} types. \\

\hl{Mixin} & A type in the OCCI model. A core component of the OCCI classification system. \\

\hl{OCCI} & Open Cloud Computing Interface \\

OCCI base type & One of \hl{Entity}, \hl{Resource}, \hl{Link} or \hl{Action}. \\

OGF & Open Grid Forum \\

\hl{Resource} & An OCCI base type. The parent type for all domain-specific resource types. \\

resource instance & An instance of a sub-type of \hl{Entity}. The OCCI
 model defines two sub-types of \hl{Entity}, the \hl{Resource} type and the
 \hl{Link} type. However, the term {\em resource instance} is defined to
 include any instance of a {\em sub-type} of \hl{Resource} or \hl{Link} as
 well. \\

Tag & A \hl{Mixin} instance with no attributes or actions defined. \\

Template & A \hl{Mixin} instance which if associated at resource instantiation
time pre-populate certain attributes. \\

type & One of the types defined by the OCCI model.  The OCCI model types are
 \hl{Category}, \hl{Kind}, \hl{Mixin}, \hl{Action}, \hl{Entity}, \hl{Resource}
 and \hl{Link}. \\

URI & Uniform Resource Identifier \\
URL & Uniform Resource Locator \\
URN & Uniform Resource Name \\
\end{tabular}

 
\section{Contributors}

We would like to thank the following people who contributed to this
document:

\begin{tabular}{l|p{2in}|p{2in}}
Name & Affiliation & Contact \\
\hline
Michael Behrens & R2AD & behrens.cloud at r2ad.com \\
Mark Carlson & Toshiba & mark at carlson.net \\
Augusto Ciuffoletti & University of Pisa & augusto.ciuffoletti at gmail.com\\
Andy Edmonds & Zhaw & andy at zhaw.ch \\
Sam Johnston & Google & samj at samj.net \\
Gary Mazzaferro & Independent &  garymazzaferro at gmail.com \\
Thijs Metsch & Intel & thijs.metsch at intel.com \\
Ralf Nyrén & Independent & ralf at nyren.net \\
Alexander Papaspyrou & Adesso & alexander at papaspyrou.name \\
Boris Parak & CESNET & parak at cesnet.cz \\
Alexis Richardson & Weaveworks & alexis.richardson at gmail.com \\
Shlomo Swidler & Orchestratus & shlomo.swidler at orchestratus.com \\
Florian Feldhaus & NetApp & florian.feldhaus at gmail.com \\
\end{tabular}

Next to these individual contributions we value the contributions from
the OCCI working group.


\section{Intellectual Property Statement}
The OGF takes no position regarding the validity or scope of any
intellectual property or other rights that might be claimed to pertain
to the implementation or use of the technology described in this
document or the extent to which any license under such rights might or
might not be available; neither does it represent that it has made any
effort to identify any such rights. Copies of claims of rights made
available for publication and any assurances of licenses to be made
available, or the result of an attempt made to obtain a general
license or permission for the use of such proprietary rights by
implementers or users of this specification can be obtained from the
OGF Secretariat.

The OGF invites any interested party to bring to its attention any
copyrights, patents or patent applications, or other proprietary
rights which may cover technology that may be required to practice
this recommendation. Please address the information to the OGF
Executive Director.


\section{Disclaimer}
This document and the information contained herein is provided on an
``As Is'' basis and the OGF disclaims all warranties, express or
implied, including but not limited to any warranty that the use of the
information herein will not infringe any rights or any implied
warranties of merchantability or fitness for a particular purpose.


\section{Full Copyright Notice}
Copyright \copyright ~Open Grid Forum (2009-2014). All Rights Reserved.

This document and translations of it may be copied and furnished to
others, and derivative works that comment on or otherwise explain it
or assist in its implementation may be prepared, copied, published and
distributed, in whole or in part, without restriction of any kind,
provided that the above copyright notice and this paragraph are
included on all such copies and derivative works. However, this
document itself may not be modified in any way, such as by removing
the copyright notice or references to the OGF or other organizations,
except as needed for the purpose of developing Grid Recommendations in
which case the procedures for copyrights defined in the OGF Document
process must be followed, or as required to translate it into
languages other than English.

The limited permissions granted above are perpetual and will not be
revoked by the OGF or its successors or assignees.


\bibliographystyle{IEEEtran}
\bibliography{references}

\end{document}
