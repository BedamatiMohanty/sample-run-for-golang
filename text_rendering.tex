\documentclass[10pt,a4paper]{article}
\usepackage[utf8]{inputenc}
\usepackage[english]{babel}
\usepackage[activate={true,nocompatibility},final,tracking=true,kerning=true,spacing=true]{microtype}
\usepackage[plainpages=false,pdfpagelabels,unicode]{hyperref}
\usepackage{fullpage}
\usepackage{graphicx}
\usepackage{fancyhdr}
\usepackage{comment}
\usepackage{occi}
\usepackage{lineno}   % adds line numbers, may be removed for non draft versions
\linenumbers          % adds line numbers, may be removed for non draft versions
\usepackage{verbatim} % adds verbatim options
\usepackage{tabularx} % adds extended tabular formatting options
\usepackage{listings}
\usepackage{color}
\definecolor{lightgray}{rgb}{.9,.9,.9}
\definecolor{darkgray}{rgb}{.4,.4,.4}
\definecolor{purple}{rgb}{0.65, 0.12, 0.82}

\lstdefinelanguage{json}{
  ndkeywords={String, Number, Boolean, Null, Object, Array},
  ndkeywordstyle=\itshape
}
\lstset{
   language=json,
   basicstyle=\footnotesize,
}

\setlength{\headheight}{13pt}
\pagestyle{fancy}

% default sans-serif
\renewcommand{\familydefault}{\sfdefault}

% no lines for headers and footers
\renewcommand{\headrulewidth}{0pt}
\renewcommand{\footrulewidth}{0pt}

% header
\fancyhf{}
\lhead{GFD-R}
\rhead{\today}

% footer
\lfoot{occi-wg@ogf.org}
\rfoot{\thepage}

% paragraphs need some space...
\setlength{\parindent}{0pt}
\setlength{\parskip}{1ex plus 0.5ex minus 0.2ex}

% some space between header and text...
\headsep 13pt

\setcounter{secnumdepth}{4}

\begin{document}

% header on first page is different
\thispagestyle{empty}

Draft \hfill Andy Edmons, ICCLab, ZHAW \\
OCCI-WG \hfill Thijs Metsch, Intel\\
\rightline {\today}

\vspace*{0.5in}

\begin{Large}
\textbf{Open Cloud Computing Interface - Text Rendering}
\end{Large}

\vspace*{0.5in}

\underline{Status of this Document}

This document is a \underline{draft} providing information to the community regarding the specification of the Open Cloud Computing Interface.

\underline{Copyright Notice}

Copyright \copyright Open Grid Forum (2015). All Rights Reserved.

\underline{Trademarks}

OCCI is a trademark of the Open Grid Forum.

\underline{Abstract}

This document, part of a document series, produced by the OCCI working
group within the Open Grid Forum (OGF), provides a high-level
definition of a Protocol and API. The document is based upon
previously gathered requirements and focuses on the scope of important
capabilities required to support modern service offerings.

\newpage
\tableofcontents
\newpage

\section{Introduction}
The Open Cloud Computing Interface (OCCI) is a RESTful Protocol and
API for all kinds of management tasks. OCCI was originally initiated
to create a remote management API for IaaS%
\footnote{Infrastructure as a Service}
model-based services, allowing for the development of interoperable tools for
common tasks including deployment, autonomic scaling and monitoring.
%
It has since evolved into an flexible API with a strong focus on
interoperability while still offering a high degree of extensibility. The
current release of the Open Cloud Computing Interface is suitable to serve many
other models in addition to IaaS, including e.g.~PaaS and SaaS.

In order to be modular and extensible the current OCCI specification is
released as a suite of complimentary documents, which together form the complete
specification.
%
The documents are divided into three categories consisting of the OCCI Core,
the OCCI Renderings and the OCCI Extensions.
%
\begin{itemize}
\item The OCCI Core specification consists of a single document defining the
 OCCI Core Model. The OCCI Core Model can be interacted with {\em
 renderings} (including associated behaviours) and expanded through {\em extensions}.
\item The OCCI Rendering specifications consist of multiple documents each
 describing a particular rendering of the OCCI Core Model. Multiple renderings can
 interact with the same instance of the OCCI Core Model and will automatically support
 any additions to the model which follow the extension rules defined in OCCI
 Core.
\item The OCCI Extension specifications consist of multiple documents each
 describing a particular extension of the OCCI Core Model. The extension documents
 describe additions to the OCCI Core Model defined within the OCCI specification
 suite.
\end{itemize}
%
The current specification consist of three documents.
Future releases of OCCI may include additional rendering and extension
specifications. The documents of the current OCCI specification suite are:

\begin{description}
\item[OCCI Core] describes the formal definition of the the OCCI Core Model
\cite{occi:core}.
\item[OCCI HTTP Rendering] defines how to interact with the OCCI Core Model using the
RESTful OCCI API \cite{occi:http_rendering}. The document defines how the OCCI Core Model can
be communicated and thus serialised using the HTTP protocol.
\item[OCCI Infrastructure] contains the definition of the OCCI Infrastructure
extension for the IaaS domain \cite{occi:infrastructure}. The document defines
additional resource types, their attributes and the actions that can be taken
on each resource type.
\end{description}


\section{Notational Conventions}
All these parts and the information within are mandatory for
implementors (unless otherwise specified). The key words "MUST", "MUST
NOT", "REQUIRED", "SHALL", "SHALL NOT", "SHOULD", "SHOULD NOT",
"RECOMMENDED", "MAY", and "OPTIONAL" in this document are to be
interpreted as described in RFC 2119 \cite{rfc2119}.

\textbf{Andy: we need to state that this document as part of the current document set,
supersedes all previous documents.}


\section{Text rendering}

This document presents the text based renderings. To be complaint, OCCI implementations MUST implement the three renderings defined in sections \ref{sec:text}, \ref{sec:header} and \ref{sec:urilist}.

The document is structured by defining based ABNFs which can then be combined into renderings which will be rendered over a protocol (e.g. HTTP) by the specific rendering definitions.

\section{ABNF Definitions}

For the following section of \ref{sec:renderings} these ABNF notations will be used. Implementations MUST hence implement the renderings according to these definitions.

\subsection{Category ABNF}

The following syntax MUST be used for \hl{Category} renderings:

\begin{verbatim}
Category             = "Category" ":" #category-value
  category-value     = term
                      ";" "scheme" "=" <"> scheme <">
                      ";" "class" "=" ( class | <"> class <"> )
                      [ ";" "title" "=" quoted-string ]
                      [ ";" "rel" "=" <"> type-identifier <"> ]
                      [ ";" "location" "=" <"> URI <"> ]
                      [ ";" "attributes" "=" <"> attribute-list <"> ]
                      [ ";" "actions" "=" <"> action-list <"> ]
  term               = (LOALPHA|DIGIT) *( LOALPHA | DIGIT | "-" | "_" )
  scheme             = URI
  type-identifier    = scheme term
  class              = "action" | "mixin" | "kind"
  attribute-list     = attribute-def
                     | attribute-def *( 1*SP attribute-def)
  attribute-def      = attribute-name
                     | attribute-name
                       "{" attribute-property *( 1*SP attribute-property ) "}"
  attribute-property = "immutable" | "required"
  attribute-name     = attr-component *( "." attr-component )
  attr-component     = LOALPHA *( LOALPHA | DIGIT | "-" | "_" )
  action-list        = action
                     | action *( 1*SP action )
  action             = type-identifier
\end{verbatim}

\subsection{Link ABNF}

The following syntax MUST be used to represent OCCI \hl{Link} type
instance references:

\begin{verbatim}
Link               = "Link" ":" #link-value
  link-value       = "<" URI-Reference ">"
                    ";" "rel" "=" <"> resource-type <">
                    [ ";" "self" "=" <"> link-instance <"> ]
                    [ ";" "category" "=" link-type
                      *( ";" link-attribute ) ]
  term             = LOALPHA *( LOALPHA | DIGIT | "-" | "_" )
  scheme           = URI
  type-identifier  = scheme term
  resource-type    = type-identifier *( 1*SP type-identifier )
  link-type        = type-identifier *( 1*SP type-identifier )
  link-instance    = URI-reference
  link-attribute   = attribute-name "=" ( token | quoted-string )
  attribute-name   = attr-component *( "." attr-component )
  attr-component   = LOALPHA *( LOALPHA | DIGIT | "-" | "_" )
\end{verbatim}

The following syntax MUST be used to represent OCCI \hl{Action}
instance references:

\begin{verbatim}
ActionLink         = "Link" ":" #link-value
  link-value       = "<" action-uri ">"
                    ";" "rel" "=" <"> action-type <">
  term             =  LOALPHA *( LOALPHA | DIGIT | "-" | "_" )
  scheme           = relativeURI
  type-identifier  = scheme term
  action-type      = type-identifier
  action-uri       = URI "?" "action=" term
\end{verbatim}

\subsection{Attribute ABNF}

\begin{verbatim}
Attribute          = "X-OCCI-Attribute" ":" #attribute-repr
  attribute-repr   = attribute-name "=" ( string | number | bool | enum_val )
  attribute-name   = attr-component *( "." attr-component )
  attr-component   = LOALPHA *( LOALPHA | DIGIT | "-" | "_" )
  string           = quoted-string
  number           = (int | float)
  int              = *DIGIT
  float            = *DIGIT "." *DIGIT
  bool             = ("true" | "false")
  enum_val         = string
\end{verbatim}

\subsection{Location ABNF}

\begin{verbatim}
Location      = "X-OCCI-Location" ":" location-value
  location-value  = URI-reference
\end{verbatim}

\section{Renderings}
\label{sec:renderings}

The renderings defined in this section will be used in the specific text rendering defined in section \ref{sec:text} and \ref{sec:header}

\subsection{Entity Instance Rendering}

Entity instances MUST be rendered according to the following definitions.

\subsubsection{Resource Instance Rendering}

A \hl{Resource} instance MUST be rendered using the following definition:

\begin{verbatim}
	resource_rendering = 1*( Category CRLF )
    	                  *( Link CRLF )
        	              *( Attribute CRLF )
\end{verbatim}

The rendering of a \hl{Resource} instance MUST represent any associated Action instances using the {\tt ActionLink CRLF}.

\paragraph{Action Invocation Rendering}

Upon an \hl{Action} invocation the client MUST send along the following definition:

\begin{verbatim}
	action_definition = 1( Category CRLF )
        	            *( Attribute CRLF )
\end{verbatim}

\subsubsection{Link Instance Rendering}

A \hl{Link} instance MUST be rendered using the following definition:

\begin{verbatim}
	link_rendering = 1*( Category CRLF )
    	              *( ActionLink CRLF )
        	          *( Attribute CRLF )
\end{verbatim}

% HERE I AM

\subsection{Category Instance Rendering}
\label{sec:format_category_instance_rendering}

A \hl{Category} instances MUST be rendered as defined below.

\subsubsection{Kind Instance Rendering}
\label{sec:format_kind}

A \hl{Kind} instance MUST be rendered as a {\tt Category CRLF}.

\subsubsection{Mixin Instance Rendering}
\label{sec:format_mixin}

A \hl{Mixin} instance MUST be rendered as a {\tt Category CRLF}.

\subsubsection{Action Instance Rendering}
\label{sec:format_action}

An \hl{Action} instance MUST be rendered as a {\tt Category CRLF}.

Note that an \hl{Action} instance MUST NOT have \hl{Link} and \hl{Action}s references.

\subsection{Entity Collection Rendering}

A collection of \hl{Resource} or \hl{Link} instances MUST be rendered as following:

\begin{verbatim}
	entity_collection_rendering = *( Location CRLF )
\end{verbatim}

\subsubsection{Resource Collection Rendering}

see above

\subsubsection{Link Collection Rendering}

see above

\subsection{Category Collection Rendering}

For the Query interface the following \hl{Category} instance rendering MUST be used:

\begin{verbatim}
	category_collection_rendering = *( Category CRLF )
\end{verbatim}

\subsubsection{Kind Collection Rendering}

see above

\subsubsection{Mixin Collection Rendering}

see above

\subsubsection{Action Collection Rendering}

see above

\subsection{Attributes Rendering}

\subsubsection{Entity Instance Attribute Rendering Specifics}

For Entity instances the following model attribute name to attribute name rendering mappings MUST be used:

\mytablefloat{
	\label{tbl:link}
	Entity attributes naming convention } {
	\begin{tabular}{ll}
		\toprule
			Attribute & Attribute name once rendered \\
		\colrule
			Entity.id & occi.core.id \\
			Entity.title & occi.core.title \\
			Resource.summary & occi.core.summary \\
			Link.target & occi.core.target \\
			Link.source & occi.core.source \\
		\botrule
	\end{tabular}
}

\subsubsection{Attribute Description Rendering}
\label{sec:format_attribute_description}

\hl{Attributes} MUST be rendered as define by the {\tt Attribute CRLF}

\section{OCCI Text Plain rendering}
\label{sec:text}
The OCCI Text plain rendering specifies a rendering of OCCI instance types in a simple text format. Using this rendering the renderings MUST be placed in the HTTP Body.

The rendering can be used to render OCCI instances independently of the
protocol being used. Thus messages can be delivered by e.g. the HTTP
protocol as specified in \cite{occi:protocol}.

The following media-types MUST be used for the OCCI Text plain rendering:

	{\tt text/occi+plain}

and

	{\tt text/plain}

Each entry in the body consists of a name followed by a colon (”:”) and the field value.

\subsection{Example}

The following example show an \hl{Entity} instance rendering using the Text plain rendering.

\begin{verbatim}
< Category: compute; \
<     scheme="http://schemas.ogf.org/occi/infrastructure#" \
<     class="kind";
< Link: </users/foo/compute/b9ff813e-fee5-4a9d-b839-673f39746096?action=start>; \
<     rel="http://schemas.ogf.org/occi/infrastructure/compute/action#start"
< X-OCCI-Attribute: occi.core.id="urn:uuid:b9ff813e-fee5-4a9d-b839-673f39746096"
< X-OCCI-Attribute: occi.core.title="My Dummy VM"
< X-OCCI-Attribute: occi.compute.architecture="x86"
< X-OCCI-Attribute: occi.compute.state="inactive"
< X-OCCI-Attribute: occi.compute.speed=1.33
< X-OCCI-Attribute: occi.compute.memory=2.0
< X-OCCI-Attribute: occi.compute.cores=2
< X-OCCI-Attribute: occi.compute.hostname="dummy"
\end{verbatim}

\section{OCCI Header Rendering}
\label{sec:header}
The following media-type MUST be used for the OCCI header Rendering:

{\tt text/occi}

While using this rendering the renderings MUST be placed in the HTTP Header. The body MUST contain the string ’OK’ on successful operations.

The HTTP header fields MUST follow the specification in RFC 7230 \cite{rfc7230}. A header field consists of a name followed by a colon (”:”) and the field value.

\textbf{Limitations}: HTTP header fields MAY appear multiple times in a HTTP request or response. In order to be OCCI compliant, the specification of multiple message-header fields according to RFC 7230 MUST be fully supported. In essence there are two valid representation of multiple HTTP header field values. A header field might either appear several times or as a single header field with a comma-separated list of field values. Due to implementation issues in many web frameworks and client libraries it is RECOMMENDED to use the comma-separated list format for best interoperability.

HTTP header field values which contain separator characters MUST be properly quoted according to RFC 7230.

Space in the HTTP header section of a HTTP request is a limited resource. By this, it is noted that many HTTP servers limit the number of bytes that can be placed in the HTTP Header area. Implementers MUST be aware of this limitation in their own implementation and take appropriate measures so that truncation of header data does NOT occur.

\subsection{Example}

The following example show an \hl{Entity} instance rendering using the Text header rendering.

\begin{verbatim}
< Category: compute; \
    scheme="http://schemas.ogf.org/occi/infrastructure#" \
    class="kind";
< Link: </users/foo/compute/b9ff813e-fee5-4a9d-b839-673f39746096?action=start>; \
    rel="http://schemas.ogf.org/occi/infrastructure/compute/action#start"
< X-OCCI-Attribute: occi.core.id="urn:uuid:b9ff813e-fee5-4a9d-b839-673f39746096", \
 occi.core.title="My Dummy VM", occi.compute.architecture="x86", \
 occi.compute.state="inactive", occi.compute.speed=1.33, \
 occi.compute.memory=2.0, occi.compute.cores=2, \
 occi.compute.hostname="dummy"
< OK
\end{verbatim}

\section{URI Listing Rendering}
\label{sec:urilist}
The following media-types MUST be used for the URI Rendering:

{\tt text/uri-list}

This rendering cannot render resource instances or Kinds or Mixins directly but just links to them. For concrete rendering of Kinds and Categories the Content-types text/occi, text/plain MUST be used. If a request is done with the text/uri-list in the Accept header, while not requesting for a Listing a Bad Request MUST be returned. Otherwise a list of resources MUST be rendered in {\\tt text/uri-list} format as defined in \cite{rfc2483}, which can be used for listing resource in collections or the name-space of the OCCI implementation.

\section{Security Considerations}
OCCI does not require that an authentication mechanism be used nor
does it require that client to service communications are secured. It
does RECOMMEND that an authentication mechanism be used and that where
appropriate, communications are encrypted using HTTP over TLS. The
authentication mechanisms that MAY be used with OCCI are those that
can be used with HTTP and TLS. For further discussion see the
appropiate section in \cite{occi:protocol}.

\section{Glossary}
\label{sec:glossary}
\begin{tabular}{l|p{12cm}}
Term & Description \\
\hline
\hl{Action} & An OCCI base type. Represent an invocable operation on a \hl{Entity} sub-type instance or collection thereof. \\

\hl{Category} & A type in the OCCI model. The parent type of \hl{Kind}. \\

\hl{Client} & An OCCI client.\\

\hl{Collection} & A set of \hl{Entity} sub-type instances all associated to a particular \hl{Kind} or \hl{Mixin} instance. \\

\hl{Entity} & An OCCI base type. The parent type of \hl{Resource} and \hl{Link}. \\

\hl{Kind} & A type in the OCCI model. A core component of the OCCI classification system. \\

\hl{Link} & An OCCI base type. A \hl{Link} instance associate one \hl{Resource} instance with another. \\

mixin & An instance of the \hl{Mixin} type associated with a {\bf resource
 instance}. The ``mixin'' concept as used by OCCI {\em only} applies to
 instances, never to \hl{Entity} types. \\

\hl{Mixin} & A type in the OCCI model. A core component of the OCCI classification system. \\

\hl{OCCI} & Open Cloud Computing Interface \\

OCCI base type & One of \hl{Entity}, \hl{Resource}, \hl{Link} or \hl{Action}. \\

OGF & Open Grid Forum \\

\hl{Resource} & An OCCI base type. The parent type for all domain-specific resource types. \\

resource instance & An instance of a sub-type of \hl{Entity}. The OCCI
 model defines two sub-types of \hl{Entity}, the \hl{Resource} type and the
 \hl{Link} type. However, the term {\em resource instance} is defined to
 include any instance of a {\em sub-type} of \hl{Resource} or \hl{Link} as
 well. \\

Tag & A \hl{Mixin} instance with no attributes or actions defined. \\

Template & A \hl{Mixin} instance which if associated at resource instantiation
time pre-populate certain attributes. \\

type & One of the types defined by the OCCI model.  The OCCI model types are
 \hl{Category}, \hl{Kind}, \hl{Mixin}, \hl{Action}, \hl{Entity}, \hl{Resource}
 and \hl{Link}. \\

URI & Uniform Resource Identifier \\
URL & Uniform Resource Locator \\
URN & Uniform Resource Name \\
\end{tabular}


\section{Contributors}

We would like to thank the following people who contributed to this
document:

\begin{tabular}{l|p{2in}|p{2in}}
Name & Affiliation & Contact \\
\hline
Michael Behrens & R2AD & behrens.cloud at r2ad.com \\
Mark Carlson & Toshiba & mark at carlson.net \\
Augusto Ciuffoletti & University of Pisa & augusto.ciuffoletti at gmail.com\\
Andy Edmonds & Zhaw & andy at zhaw.ch \\
Sam Johnston & Google & samj at samj.net \\
Gary Mazzaferro & Independent &  garymazzaferro at gmail.com \\
Thijs Metsch & Intel & thijs.metsch at intel.com \\
Ralf Nyrén & Independent & ralf at nyren.net \\
Alexander Papaspyrou & Adesso & alexander at papaspyrou.name \\
Boris Parak & CESNET & parak at cesnet.cz \\
Alexis Richardson & Weaveworks & alexis.richardson at gmail.com \\
Shlomo Swidler & Orchestratus & shlomo.swidler at orchestratus.com \\
Florian Feldhaus & NetApp & florian.feldhaus at gmail.com \\
\end{tabular}

Next to these individual contributions we value the contributions from
the OCCI working group.


\section{Intellectual Property Statement}
The OGF takes no position regarding the validity or scope of any
intellectual property or other rights that might be claimed to pertain
to the implementation or use of the technology described in this
document or the extent to which any license under such rights might or
might not be available; neither does it represent that it has made any
effort to identify any such rights. Copies of claims of rights made
available for publication and any assurances of licenses to be made
available, or the result of an attempt made to obtain a general
license or permission for the use of such proprietary rights by
implementers or users of this specification can be obtained from the
OGF Secretariat.

The OGF invites any interested party to bring to its attention any
copyrights, patents or patent applications, or other proprietary
rights which may cover technology that may be required to practice
this recommendation. Please address the information to the OGF
Executive Director.


\section{Disclaimer}
This document and the information contained herein is provided on an
``As Is'' basis and the OGF disclaims all warranties, express or
implied, including but not limited to any warranty that the use of the
information herein will not infringe any rights or any implied
warranties of merchantability or fitness for a particular purpose.


\section{Full Copyright Notice}
Copyright \copyright ~Open Grid Forum (2009-2014). All Rights Reserved.

This document and translations of it may be copied and furnished to
others, and derivative works that comment on or otherwise explain it
or assist in its implementation may be prepared, copied, published and
distributed, in whole or in part, without restriction of any kind,
provided that the above copyright notice and this paragraph are
included on all such copies and derivative works. However, this
document itself may not be modified in any way, such as by removing
the copyright notice or references to the OGF or other organizations,
except as needed for the purpose of developing Grid Recommendations in
which case the procedures for copyrights defined in the OGF Document
process must be followed, or as required to translate it into
languages other than English.

The limited permissions granted above are perpetual and will not be
revoked by the OGF or its successors or assignees.


\bibliographystyle{IEEEtran}
\bibliography{references}

\end{document}
