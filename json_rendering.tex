\documentclass[10pt,a4paper]{article}
\usepackage[utf8]{inputenc}
\usepackage{fullpage}
\usepackage{graphicx}
\usepackage{fancyhdr}
\usepackage{comment}
\usepackage{occi}
\usepackage{lineno} % adds line numbers, may be removed for non draft versions
\linenumbers        % adds line numbers, may be removed for non draft versions
\usepackage{minted} % code highlighting
\setlength{\headheight}{13pt}
\pagestyle{fancy}

% default sans-serif
\renewcommand{\familydefault}{\sfdefault}

% no lines for headers and footers
\renewcommand{\headrulewidth}{0pt}
\renewcommand{\footrulewidth}{0pt}

% header
\fancyhf{}
\lhead{GWD-R}
\rhead{\today}

% footer
\lfoot{occi-wg@ogf.org}
\rfoot{\thepage}

% paragraphs need some space...
\setlength{\parindent}{0pt}
\setlength{\parskip}{1ex plus 0.5ex minus 0.2ex}

% some space between header and text...
\headsep 13pt

\setcounter{secnumdepth}{4}

\begin{document}

% header on first page is different
\thispagestyle{empty}

GWD-R \hfill Ralf Nyrén \\
OCCI-WG \hfill Florian Feldhaus, GWDG\\
\rightline {February 25, 2011}\\
\rightline {Updated: \today}

\vspace*{0.5in}

\begin{Large}
\textbf{Open Cloud Computing Interface - JSON Rendering}
\end{Large}

\vspace*{0.5in}

\underline{Status of this Document}

This document provides information to the community regarding the
specification of the Open Cloud Computing Interface. Distribution is
unlimited.

\underline{Copyright Notice}

Copyright \copyright Open Grid Forum (2012). All Rights Reserved.

\underline{Trademarks}

OCCI is a trademark of the Open Grid Forum.

\underline{Abstract}

This document, part of a document series, produced by the OCCI working
group within the Open Grid Forum (OGF), provides a high-level
definition of a Protocol and API. The document is based upon
previously gathered requirements and focuses on the scope of important
capabilities required to support modern service offerings.

\underline{Comments}
\newcommand{\ralf}[1]{\textcolor{red}{RN: #1}}
\newcommand{\andy}[1]{\textcolor{green}{AE: #1}}

\newpage
\tableofcontents
\newpage

\section{Introduction}
%The Open Cloud Computing Interface (OCCI) is a RESTful Protocol and
API for all kinds of management tasks. OCCI was originally initiated
to create a remote management API for IaaS%
\footnote{Infrastructure as a Service}
model-based services, allowing for the development of interoperable tools for
common tasks including deployment, autonomic scaling and monitoring.
%
It has since evolved into an flexible API with a strong focus on
interoperability while still offering a high degree of extensibility. The
current release of the Open Cloud Computing Interface is suitable to serve many
other models in addition to IaaS, including e.g.~PaaS and SaaS.

In order to be modular and extensible the current OCCI specification is
released as a suite of complimentary documents, which together form the complete
specification.
%
The documents are divided into three categories consisting of the OCCI Core,
the OCCI Renderings and the OCCI Extensions.
%
\begin{itemize}
\item The OCCI Core specification consists of a single document defining the
 OCCI Core Model. The OCCI Core Model can be interacted with {\em
 renderings} (including associated behaviours) and expanded through {\em extensions}.
\item The OCCI Rendering specifications consist of multiple documents each
 describing a particular rendering of the OCCI Core Model. Multiple renderings can
 interact with the same instance of the OCCI Core Model and will automatically support
 any additions to the model which follow the extension rules defined in OCCI
 Core.
\item The OCCI Extension specifications consist of multiple documents each
 describing a particular extension of the OCCI Core Model. The extension documents
 describe additions to the OCCI Core Model defined within the OCCI specification
 suite.
\end{itemize}
%
The current specification consist of three documents.
Future releases of OCCI may include additional rendering and extension
specifications. The documents of the current OCCI specification suite are:

\begin{description}
\item[OCCI Core] describes the formal definition of the the OCCI Core Model
\cite{occi:core}.
\item[OCCI HTTP Rendering] defines how to interact with the OCCI Core Model using the
RESTful OCCI API \cite{occi:http_rendering}. The document defines how the OCCI Core Model can
be communicated and thus serialised using the HTTP protocol.
\item[OCCI Infrastructure] contains the definition of the OCCI Infrastructure
extension for the IaaS domain \cite{occi:infrastructure}. The document defines
additional resource types, their attributes and the actions that can be taken
on each resource type.
\end{description}


\section{Notational Conventions}
All these parts and the information within are mandatory for
implementors (unless otherwise specified). The key words "MUST", "MUST
NOT", "REQUIRED", "SHALL", "SHALL NOT", "SHOULD", "SHOULD NOT",
"RECOMMENDED", "MAY", and "OPTIONAL" in this document are to be
interpreted as described in RFC 2119 \cite{rfc2119}.

\textbf{Andy: we need to state that this document as part of the current document set,
supersedes all previous documents.}


%The following terms \cite{rfc3986} are used when referring to URI
%components:
%
%\begin{verbatim}
% http://example.com:8080/over/there?action=stop#xyz
% \__/   \______________/\_________/ \_________/ \_/
%  |           |            |            |        |
%scheme     authority       path        query   fragment
%\end{verbatim}

\section{OCCI JSON Rendering}
{\em TBD: intro, JSON Rendering is RESTful, can happily co-exist with the
existing HTTP Rendering, etc}

\section{Namespace}

The JSON Rendering provides a rendering (i.e. serialisation) of the OCCI 
Core model into the URL hierarchy by binding
\hl{Kind} and \hl{Mixin} instances to unique URL paths. Such a URL path is called
the {\em location} of the \hl{Kind} or \hl{Mixin}.
A provider is free to choose the {\em location} as long as it is unique
within the service provider's URL namespace.
For example, the \hl{Kind} instance%
\footnote{\tt http://schemas.ogf.org/occi/infrastructure\#compute}
for the \hl{Compute} type may be bound to {\tt /my/occi/api/compute/}.

A \hl{Kind} instance whose associated type cannot be instantiated MUST NOT be
bound to an URL path. This applies to the \hl{Kind} instance for OCCI Entity.

\subsection{Bound and Unbound Paths}

Since a limited set of URL paths are bound to \hl{Kind} and \hl{Mixin}
instances the URL hierarchy consists of both {\em bound} and {\em unbound}
paths.
A bound URL path is the {\em location} of a \hl{Kind} or \hl{Mixin} collection.

An unbound URL path MAY represent the union of all \hl{Kind} and \hl{Mixin}
collection ``below'' the unbound path.
\ralf{FIXME: Should this be a MUST instead?}
\andy{Q: Do these apply for instances also?}

% TODO move to OCCI Core

%\subsection{Unique identifiers}
%
%In the OCCI JSON Rendering the {\tt occi.core.id} attribute MUST be a UUID
%\cite{rfc4122} in canonical form, for example {\tt 30782156-c5e1-43c2-99fd-7cf46bc2daa7}.
%
%A resource instance MUST exist in the collection of its associated \hl{Kind}
%and \hl{Mixin}s. Concatenation of a resource instance's \hl{Kind} or \hl{Mixin}
%location together with its {\tt occi.core.id} MUST yield a valid URL from which
%the resource instance can be retrieved.
%
%\paragraph*{Example 1}
%A resource instance of the \hl{Network} type has been assigned a unique identifier
%so that {\tt occi.core.id} is {\tt 30782156-c5e1-43c2-99fd-7cf46bc2daa7}. The \hl{Kind} of
%the \hl{Network} type is bound to location {\tt /network/}.
%
%A request to {\tt /network/30782156-c5e1-43c2-99fd-7cf46bc2daa7} MUST then yield a
%representation of the \hl{Network} instance.
%
%\paragraph*{Example 2}
%The same resource instance as in example 1 is also associated with the
%\hl{IPNetwork} \hl{Mixin} which is bound to location {\tt /ipnetwork/}.
%
%A request to {\tt /ipnetwork/30782156-c5e1-43c2-99fd-7cf46bc2daa7} MUST yield a
%representation of the same \hl{Network} instance as in example 1.

\section{JSON Format}
\label{sec:json_format}
The OCCI JSON Rendering consists of a JSON object holding information on the 
OCCI core objects kind, mixin, action, link and resource.

The following media-type has been assigned to the OCCI JSON Rendering:

{\tt application/occi+json}


%All four formats use a top-level JSON object with a different set of object
%members. The details of these structures are described in the subsections
%below.
%\begin{description}
%\item[\tt application/occi-entity+json]
%A single OCCI Resource or OCCI Link instance.
%See section~\ref{sec:format_entity}.
%
%\item[\tt application/occi-collection+json]
%A collection of OCCI Resource and OCCI Link instances.
%See section~\ref{sec:format_collection}.
%
%\item[\tt application/occi-action+json]
%Invocation of an OCCI Action.
%See section~\ref{sec:format_action}.
%
%\item[\tt application/occi-discovery+json]
%Format used by the OCCI discovery mechanism which comprise the OCCI type system
%\cite{occi:core}.
%See section~\ref{sec:format_discovery}.
%\end{description}

%An OCCI client MUST specify the media-type whenever request payload is
%submitted. Likewise an OCCI server MUST specify the media-type of the response
%payload. The media-type MUST be specified in the HTTP Content-Type header as specified
%in RFC2616 \cite{rfc2616}.

\subsection{Resource Instance Format}
\label{sec:format_resource}

A {\em resource instance} is a sub-type of OCCI Entity which has been instantiated.
OCCI Resource and OCCI Link are the top-most sub-types of OCCI Entity.
OCCI Entity itself cannot be instantiated.

The resource instance format consists of a JSON object as shown in the
following example. Section \ref{sec:example_resource} contains a detailed example.
Table~\ref{tbl:format_resource} defines the object members.
\begin{minted}{json}
{
    "resources": [
        {
            "kind": "...",
            "mixins": [ "...", "..." ],
            "attributes": { },
            "actions": [ { }, { } ],
            "id": "...",
            "title": "...",
            "summary": "...",
            "links": [ { }, { } ],
        }
    ]
}
\end{minted}
\mytablefloat{
    \label{tbl:format_resource}
    Resource instances are rendered using the
    {\tt application/occi+json} format which consists of a JSON object with name {\em resources}
    containing an array of JSON objects with the following entries.
    } {
    \begin{tabular}{llp{5.0cm}p{3.0cm}}
    \toprule
    Object member & JSON type & Description & Necessity \\
    \colrule
    kind & string & Kind identifier & Mandatory \\

    mixins & array of strings & List of Mixin identifiers &
    Mandatory if resource has mixins \\

    attributes & object & Instance attributes & Mandatory if resource has attributes \\

    actions & array of objects & Actions applicable to the resource instance as defined in~\ref{tbl:format_action} &
    Mandatory if resource has applicable actions \\
    
    id & string & UUID of the resource & Mandatory\\
        
    title & string & Title of the resource & Optional\\
    
    summary & string & Summary describing the resource & Optional\\
    
    links & array of objects & Associated OCCI Links as defined in~\ref{tbl:format_link} &
    Mandatory if resource has links\\
    \botrule
    \end{tabular}
}
\subsection{Link Instance Format}
\label{sec:format_link}

The link instance format consists of a JSON object as shown in the
following example. Section \ref{sec:example_link} contains a detailed example.
Table~\ref{tbl:format_link} defines the object members.
\begin{minted}{json}
{
    "links": [
        {
            "kind": "...",
            "mixins": [ "...", "..." ],
            "attributes": { },
            "actions": [ { }, { } ],
            "id": "...",
            "title": "...",
            "target": "...",
            "source": "..."
        }
    ]
}
\end{minted}
\mytablefloat{
    \label{tbl:format_link}
    Link instances are rendered using the
    {\tt application/occi+json} format which consists of a JSON object with name {\em links}
    containing an array of JSON objects with the following entries.
    } {
    \begin{tabular}{llp{5.0cm}p{3.0cm}}
    \toprule
    Object member & JSON type & Description & Necessity \\
    \colrule
    kind & string & Kind identifier & Mandatory \\

    mixins & array of strings & List of Mixin identifiers &
    Mandatory if resource has mixins \\

    attributes & object & Instance attributes & Mandatory if resource has attributes \\

    actions & array of objects & Actions applicable to the resource instance as defined in~\ref{tbl:format_action} &
    Mandatory if resource has applicable actions \\
    
    id & string & UUID of the link & Mandatory\\
        
    title & string & Title of the link & Optional\\
        
    target & string & Absolute location of target resource & Mandatory \\
    
    source & string & Absolute location of source resource & Mandatory unless rendered within the source resource\\
    \botrule
    \end{tabular}
}

%\subsection{Collection format}
%\label{sec:format_collection}
%
%A collection of resource instances is rendered in the
%{\tt application/occi-collection+json} format.
%A collection may consist of resource instances of different types.
%
%Every \hl{Kind} and \hl{Mixin} instance represent a collection of its
%associated resource instances. For example the \hl{Kind} instance for
%the \hl{Compute} type,
%{\tt http://schemas.ogf.org/occi/infrastructure\#compute}, automatically
%represent a collection of all \hl{Compute} instances \cite{occi:core}.
%The collection format MAY be used to render a union of several collections
%as one big combined collection.
%
%The collection format consists of a JSON object as shown in the example below.
%Table~\ref{tbl:format_collection} define the object members.
%\begin{verbatim}
%{
%  collection: [ ... ],
%  size: 1004,
%  limit: 50,
%  next: "http://example.com/compute/?marker=25c46a34-3caa-4713-806b-6abb9593ba2d&limit=50"
%}
%\end{verbatim}
%\mytablefloat{
%    \label{tbl:format_collection}
%    A collection of resource instances is rendered in the
%    {\tt application/occi-collection+json} format.
%    The format consists of a JSON object with the following members.
%    } {
%    \begin{tabular}{lllp{3.0cm}}
%    \toprule
%    Object member & JSON type & Description & Applicable to \\
%    \colrule
%    collection & array of objects & List of resource instances & Request/response \\
%
%    size & integer & Total size of the collection & Response {\em only} \\
%
%    limit & integer & Effective page size limit  & Response {\em only} \\
%
%    next & string & URL of the next page  & Response {\em only} \\
%    \botrule
%    \end{tabular}
%}
%The {\tt collection} array consists of JSON objects in {\em single resource
%instance format} described in section~\ref{sec:format_entity}.
%
%The collection format has built-in support for pagination. Pagination allows a
%subset of a collection to be transferred to reduce server load and client
%processing.
%An OCCI server MUST specify the {\tt next} member whenever a paginated collection
%is returned. See section~\ref{sec:collection:pagination}.
%%
%\subsection{Action invocation format}
%\label{sec:format_action}
%
%An action invocation is rendered in the {\tt application/occi-action+json}
%format. An action is an invokable operation on a resource instance and is
%identified by a OCCI Category instance.
%
%The action invocation format consists of a JSON object as shown in the example
%below. Table~\ref{tbl:format_action} define the object members.
%Action invocations are only relevant to client requests.
%\begin{verbatim}
%{
%  "category": "http://schemas.ogf.org/occi/infrastructure/compute/action#stop",
%  "attributes": {
%    "method": "graceful"
%  }
%}
%\end{verbatim}
%\mytablefloat{
%    \label{tbl:format_action}
%    An action represent an invocable operation on a resource instance.
%    The {\tt application/occi-action+json} format is used to render an action
%    invocation. The format consists of a JSON object with the following
%    members.
%    } {
%    \begin{tabular}{lll}
%    \toprule
%    Object member & JSON type & Description \\
%    \colrule
%    category & string & Category identifier \\
%    attributes & object & Action attributes \\
%    \botrule
%    \end{tabular}
%}

\subsection{Kind Format}
\label{sec:format_kind}

An OCCI kind is used to describe a OCCI entity and cannot itself be 
instantiated. OCCI kinds can only be queried through the query interface 
of an OCCI server to get a complete description of a specific OCCI entity sub-type.

The kind format consists of a JSON object as shown in the
following example. Section \ref{sec:example_kind} contains a detailed example.
Table~\ref{tbl:format_kind} defines the top-level object members.

\mytablefloat{
    \label{tbl:format_kind}
    Kinds are rendered using the
    {\tt application/occi+json} format which consists of a JSON object with name {\em kinds}
    containing an array of JSON objects with the following entries.
    } {
    \begin{tabular}{llll}
    \toprule
    Object member & JSON type & Description & Necessity\\
    \colrule
    term & string & Unique identifier within the categorisation scheme & Mandatory\\
    scheme & string & Categorisation scheme & Mandatory\\
    title & string & Title of the kind & Mandatory\\
    attributes & object & Attribute description, see ~\ref{tbl:format_attribute} & Mandatory if kind has attributes\\
    related & array of strings & List containing the related ``parent'' \hl{Kind} instance & Mandatory if kind is related to another kind\\
    actions & array of strings & List of action identifiers & Mandatory if kind has actions\\
    location & string & Relative URL bound to the \hl{Kind} instance & Mandatory\\
    \botrule
    \end{tabular}
}

\begin{minted}{json}
{
    "kinds": [
        {
            "term": "...",
            "scheme": "...",
            "title": "...",
            "attributes": { },
            "actions": [ "...", "..." ],
            "related": [ "...", "..." ],
            "location": "..."
        }
    ]
}
\end{minted}

\subsection{Mixin Format}
\label{sec:format_mixin}

An OCCI mixin can be used to extend OCCI entities and cannot itself be 
instantiated. OCCI mixins can be queried through the query interface 
of an OCCI server and also be created by a user, where permitted.

The mixin format consists of a JSON object as shown in the
following example. Section \ref{sec:example_kind} contains a detailed example.
Table~\ref{tbl:format_mixin} defines the top-level object members.

\mytablefloat{
    \label{tbl:format_mixin}
    Mixins are rendered using the
    {\tt application/occi+json} format which consists of a JSON object with name {\em mixins}
    containing an array of JSON objects with the following entries.
    } {
    \begin{tabular}{llll}
    \toprule
    Object member & JSON type & Description & Necessity\\
    \colrule
    term & string & Unique identifier within the categorisation scheme & Mandatory\\
    scheme & string & Categorisation scheme & Mandatory\\
    title & string & Title of the mixin & Mandatory\\
    attributes & object & Attribute description, see ~\ref{tbl:format_attribute} & Mandatory if mixin has attributes\\
    related & array of strings & List containing the related ``parent'' \hl{Mixin} instance & Mandatory if mixin is related to other mixins\\
    actions & array of strings & List of action identifiers & Mandatory if mixin has actions\\
    location & string & Relative URL bound to the \hl{Kind} instance & Mandatory\\
    \botrule
    \end{tabular}
}

\begin{minted}{json}
{
    "mixins": [
        {
            "term": "...",
            "scheme": "...",
            "title": "...",
            "attributes": { },
            "actions": [ "...", "..." ],
            "related": [ "...", "..." ],
            "location": "..."
        }
    ]
}
\end{minted}

\subsection{Action Format}
\label{sec:format_action}

An OCCI action can be used to trigger specific actions on an OCCI entity and cannot itself be 
instantiated. OCCI actions can only be queried through the query interface 
of an OCCI server.

The action format consists of a JSON object as shown in the
following example.
Table~\ref{tbl:format_action} defines the top-level object members.

\mytablefloat{
    \label{tbl:format_action}
    Actions are rendered using the
    {\tt application/occi+json} format which consists of a JSON object with name {\em actions}
    containing an array of JSON objects with the following entries.
    } {
    \begin{tabular}{llll}
    \toprule
    Object member & JSON type & Description & Necessity\\
    \colrule
    term & string & Unique identifier within the categorisation scheme & Mandatory\\
    scheme & string & Categorisation scheme & Mandatory\\
    title & string & Title of the mixin & Optional\\
    attributes & object & Attribute description, see ~\ref{tbl:format_attribute} & Mandatory if action has attributes\\
    \botrule
    \end{tabular}
}

\begin{minted}{json}
{
    "actions": [
        {
            "term": "...",
            "scheme": "...",
            "title": "...",
            "attributes": { }
        }
	]
}
\end{minted}

\subsection{Attribute Description Format}
\label{sec:format_attribute_description}

Attribute descriptions of OCCI Categories are rendered as JSON objects. 
The dots of the attribute names define a hierarchy. This hierarchy is reflected 
by JSON objects within the higher layer JSON object or within the top level 
JSON object with name {\em attributes}. The last part of the attribute name
 hierarchy includes the properties-object pairs of the attribute as defined 
 in table~\ref{tbl:format_attribute_description}

\mytablefloat{
    \label{tbl:format_attribute_description}
    The attribute-properties object has the members defined in this table.
    %
    All attribute properties are optional and the table shows which property default value
    an OCCI client MUST assume if a particular property is unspecified.
    } {
    \begin{tabular}{llll}
    \toprule
    Object member & JSON type & Description & Default \\
    \colrule
    mutable & boolean & Attribute mutability & true \\
    required & boolean & Whether the attribute MUST be specified at resource instantiation & false \\
    type & string & Enum \{string, number, boolean\}
    & string \\
    pattern & string & Posix Extended Regular Expression as defined in \cite{iso9945:2009}. For interoperability reasons, POSIX character classes like e.g. [:alpha:] should not be used. & .* \\
    minimum & number & If type is a number, then minimum defines the lowest number allowed. If type is a string, then minimum defines the minimal length of the string. \\
    maximum & If type is a number, then maximum defines the highest number allowed. If type is a string, then maximum defines the maximal length of the string.\\
    default & string, number or boolean & Attribute default when not specified by client. & null \\
    \botrule
    \end{tabular}
}
\begin{minted}{json}
{
    "attributes": {
        "...": {
            "mutable": true,
            "required": false,
            "type": "string",
            "pattern": ".*",
            "minimum": 1,
            "maximum": 65535,
            "default": null
        }
    }
}
\end{minted}

\section{Detailed Examples}
\label{sec:examples}

\subsection{Resource Instance Format Example}
\label{sec:example_resource}

\begin{minted}{json}
{
    "resources": [
        {
            "kind": "http: //schemas.ogf.org/occi/infrastructure#compute",
            "mixins": [
                "http: //schemas.opennebula.org/occi/infrastructure#my_mixin",
                "http: //schemas.other.org/occi#my_mixin"
            ],
            "attributes": {
                "occi": {
                    "compute": {
                        "speed": 2,
                        "memory": 4,
                        "cores": 2
                    }
                },
                "org": {
                    "other": {
                        "occi": {
                            "my_mixin": {
                                "my_attribute": "my_value"
                            }
                        }
                    }
                }
            },
            "actions": [
                {
                    "title": "Start My Server",
                    "href": "/compute/996ad860-2a9a-504f-8861-aeafd0b2ae29?action=start",
                    "category": "http://schemas.ogf.org/occi/infrastructure/compute/action#start"
                }
            ],
            	"id": "996ad860-2a9a-504f-8861-aeafd0b2ae29",
		    "title": "Compute resource",
		    "summary": "This is a compute resource",
		    "links": [
		        {
		            "target": "http://myservice.tld/storage/59e06cf8-f390-5093-af2e-3685be593a25",
		            "kind": "http://schemas.ogf.org/occi/infrastructure#storagelink",
		            "attributes": {
		                "occi": {
		                    "storagelink": {
		                        "deviceid": "ide:0:1"
		                    }
		                }
		            },
		            "id": "391ada15-580c-5baa-b16f-eeb35d9b1122",
		            "title": "My disk"
		        }
		    ]
        }
    ]
}
\end{minted}

\subsection{Link Instance Format Example}
\label{sec:example_link}

\begin{minted}{json}
{
    "links": [
        {
            "kind": "http://schemas.ogf.org/occi/infrastructure#networkinterface",
            "mixins": [
                "http://schemas.ogf.org/occi/infrastructure/networkinterface#ipnetworkinterface"
            ],
            "attributes": {
                "occi": {
                    "infrastructure": {
                        "networkinterface": {
                            "interface": "eth0",
                            "mac": "00:80:41:ae:fd:7e",
                            "address": "192.168.0.100",
                            "gateway": "192.168.0.1",
                            "allocation": "dynamic"
                        }
                    }
                }
            },
            "actions": [
                {
                    "title": "Disable networkinterface",
                    "href": "/networkinterface/22fe83ae-a20f-54fc-b436-cec85c94c5e8?action=up",
                    "category": "http: //schemas.ogf.org/occi/infrastructure/networkinterface/action#up"
                }
            ],
            "id": "22fe83ae-a20f-54fc-b436-cec85c94c5e8",
		    "title": "My network interface",
            "target": "http://myservice.tld/network/b7d55bf4-7057-5113-85c8-141871bf7635",
            "source": "http://myservice.tld/compute/996ad860-2a9a-504f-8861-aeafd0b2ae29"
        }
    ]
}
\end{minted}

\subsection{Kind Format Example}
\label{sec:example_kind}

\begin{minted}{json}
{
    "kinds": [
        {
            "term": "compute",
            "scheme": "http://schemas.ogf.org/occi/infrastructure#",
            "title": "Compute Resource",
            "related": [
                "http://schemas.ogf.org/occi/core#resource"
            ],
            "attributes": {
                "occi": {
                    "compute": {
                        "hostname": {
                            "mutable": true,
                            "required": false,
                            "type": "string",
                            "pattern": "(([a-zA-Z0-9]|[a-zA-Z0-9][a-zA-Z0-9\\-]*[a-zA-Z0-9])\\.)*",
                            "minimum": "1",
                            "maximum": "255"
                        },
                        "state": {
                            "mutable": false,
                            "required": false,
                            "type": "string",
                            "pattern": "inactive|active|suspended|failed",
                            "default": "inactive"
                        }
                    }
                }
            },
            "actions": [
                "http://schemas.ogf.org/occi/infrastructure/compute/action#start",
                "http://schemas.ogf.org/occi/infrastructure/compute/action#stop",
                "http://schemas.ogf.org/occi/infrastructure/compute/action#restart",
                "http://schemas.ogf.org/occi/infrastructure/compute/action#suspend"
            ],
            "location": "/compute/"
        }
    ]
}
\end{minted}

\subsection{Mixin Format Example}
\label{sec:example_mixin}

\begin{minted}{json}
{
    "mixins": [
        {
            "term": "medium",
            "scheme": "http://example.com/template/resource#",
            "title": "Medium VM",
            "related": [
                "http://schemas.ogf.org/occi/infrastructure#resource_tpl"
            ],
            "attributes": {
                "occi": {
                    "compute": {
                        "speed": {
                            "type": "number",
                            "default": 2.8
                        }
                    }
                }
            },
            "location": "/template/resource/medium/"
        }
    ]
}
\end{minted}

\subsection{Action Format Example}
\label{sec:example_action}

\begin{minted}{json}
{
    "actions": [
        {
            "term": "stop",
            "scheme": "http://schemas.ogf.org/occi/infrastructure/compute/action#",
            "title": "Stop Compute instance",
            "attributes": {
                "method": {
                    "mutable": true,
                    "required": false,
                    "type": "string",
                    "pattern": "graceful|acpioff|poweroff",
                    "default": "poweroff"
                }
            }
        }
    ]
}
\end{minted}

%\section{HTTP methods applied to resource instance URLs}
%
%This section describes the HTTP methods used to retrieve and manipulate
%individual resource instances. Each HTTP method described is assumed to operate
%on an URL referring to a single element in a collection, an URL such as the
%following:
%\begin{verbatim}
%  http://example.com/compute/012d2b48-c334-47f2-9368-557e75249042
%\end{verbatim}
%
%An OCCI server MUST use the request processing defined by this document
%whenever the Content-Type of a request use one of the media-types defined in
%section~\ref{sec:json_format}.
%%
%In order to receive a response defined by the OCCI JSON Rendering a client MUST
%include one or several of the JSON media-types%
%\footnote{See section~\ref{sec:json_format}.}
%in the Accept header.
%
%At the beginning of every ``Client request'' and ``Server response''
%sub-section below the required Accept and Content-Type headers are defined. A
%client MAY supply all four JSON media-types in the Accept header in which case
%the server MUST choose the most suitable format.
%
%\subsection{GET resource instance}
%The HTTP GET method retrieves the JSON representation of an OCCI resource
%instance.
%
%\subsubsection{Client GET request}
%\begin{description}
%\item[Accept] {\tt application/occi-entity+json}
%\end{description}
%The body of the HTTP GET request MUST be empty.
%\begin{verbatim}
%GET /compute/012d2b48-c334-47f2-9368-557e75249042 HTTP/1.1
%Host: example.com
%Accept: application/occi-entity+json
%User-Agent: occi-client/x.x OCCI/1.1
%\end{verbatim}
%
%\subsubsection{Server GET response}
%\begin{description}
%\item[Content-Type] {\tt application/occi-entity+json}
%\end{description}
%The body of the HTTP GET response MUST contain a rendering of the resource
%instance in the single-resource-instance format, see
%section~\ref{sec:format_entity}.
%\begin{verbatim}
%HTTP/1.1 200 OK
%Server: occi-server/x.x OCCI/1.1
%Content-Type: application/occi-entity+json; charset=utf-8
%
%{
%  "kind": "http://schemas.ogf.org/occi/infrastructure#compute",
%  "mixins": [ ... ],
%  "actions": [ ... ],
%  "links": [ ... ],
%  "attributes": { ... }
%}
%\end{verbatim}
%
%\subsection{PUT resource instance}
%The HTTP PUT method creates or replaces the resource instance at the specified
%URL. Since the UUID of the resource instance is supplied in the request URL an OCCI
%server MAY refuse to create a new instance.
%If specified the {\tt occi.core.id} attribute MUST be identical to the UUID in
%the request URL.
%
%\subsubsection{Client PUT request}
%\begin{description}
%\item[Accept] {\tt application/occi-entity+json}
%\item[Content-Type] {\tt application/occi-entity+json}
%\end{description}
%The full representation of the resource instance MUST be supplied in the HTTP
%body of the request. The request body MUST use the single-resource-instance
%format defined in section~\ref{sec:format_entity}.
%
%The object member {\tt links} MUST NOT be supplied by a client. Removing a set
%of OCCI Links breaks the idempotence of a PUT request. Any Links present for an
%existing OCCI Resource MUST be left intact. A server MUST refuse a request
%containing the {\tt links} member.
%
%\begin{verbatim}
%PUT /compute/012d2b48-c334-47f2-9368-557e75249042 HTTP/1.1
%Host: example.com
%Accept: application/occi-entity+json
%User-Agent: occi-client/x.x OCCI/1.1
%Content-Type: application/occi-entity+json; charset=utf-8
%
%{
%  "kind": "http://schemas.ogf.org/occi/infrastructure#compute",
%  "mixins": [ ... ],
%  "attributes": { ... }
%}
%\end{verbatim}
%
%\subsubsection{Server PUT response}
%\begin{description}
%\item[Content-Type] {\tt application/occi-entity+json}
%\end{description}
%Upon success an OCCI server MUST return HTTP status code 200 and a complete
%JSON representation of the created/replaced resource instance in
%single-resource-instance format.
%The response MUST be identical%
%\footnote{Provided the resource instance was not changed in the meantime.}
%to that of a subsequent GET request of the same URL.
%
%\begin{verbatim}
%HTTP/1.1 200 OK
%Server: occi-server/x.x OCCI/1.1
%Content-Type: application/occi-entity+json; charset=utf-8
%
%{
%  "kind": "http://schemas.ogf.org/occi/infrastructure#compute",
%  "mixins": [ ... ],
%  "actions": [ ... ],
%  "links": [ ... ],
%  "attributes": { ... }
%}
%\end{verbatim}
%
%
%\subsection{POST resource instance (action)}
%There are two methods to invoke an OCCI Action using the JSON Rendering.
%\begin{enumerate}
%\item Supply the query parameter ``action'' together with the request. The value
%of ``action'' MUST be the {\tt term} of the action \hl{Category}.
%\item Specify {\tt application/occi-action+json} in the Content-Type header
%and supply a request payload formatted according to section~\ref{sec:format_action}.
%In order to specify action attributes this method MUST be used.
%\end{enumerate}
%An OCCI Client MAY combine the two methods if the ``action'' parameter's value
%is equal to the \hl{Category} {\tt term} in the body.
%
%\subsubsection{Client POST action request}
%\begin{description}
%\item[Accept] {\tt application/occi-entity+json}
%\item[Content-Type] {\tt application/occi-action+json}
%\end{description}
%The example shows the combined method.
%\begin{verbatim}
%POST /compute/012d2b48-c334-47f2-9368-557e75249042?action=stop HTTP/1.1
%Host: example.com
%Accept: application/occi-entity+json
%User-Agent: occi-client/x.x OCCI/1.1
%Content-Type: application/occi-action+json; charset=utf-8
%
%{
%  "category": "http://schemas.ogf.org/occi/infrastructure/compute/action#stop",
%  "attributes": {
%    "method": "graceful"
%  }
%}
%\end{verbatim}
%
%\subsubsection{Server POST action response}
%\begin{description}
%\item[Content-Type] {\tt application/occi-entity+json}
%\end{description}
%If the request Accept header contains {\tt application/occi-entity+json} the
%server MAY return status code 200 and a full representation of the resource instance.
%Otherwise the server MUST return status code 204 and no response payload.
%\begin{verbatim}
%HTTP/1.1 204 OK
%Server: occi-server/x.x OCCI/1.1
%\end{verbatim}
%
%\subsection{POST resource instance}
%\begin{description}
%\item[Content-Type] {\tt application/occi-entity+json}
%\end{description}
%\ralf{This would imply a partial update of the resource instance. While it is
%easy to supply only the attributes to be updated the question is if there are any
%valid use cases for partial updates using JSON?}
%
%\subsection{DELETE resource instance}
%The HTTP DELETE method destroys a resource instance and any OCCI Links
%associated with an OCCI Resource.
%
%\subsubsection{Client DELETE request}
%\begin{description}
%\item[Content-Type] {\tt application/occi-entity+json}
%\end{description}
%\begin{verbatim}
%DELETE /compute/012d2b48-c334-47f2-9368-557e75249042 HTTP/1.1
%Host: example.com
%Accept: application/occi-entity+json
%\end{verbatim}
%
%\subsubsection{Server DELETE response}
%\begin{verbatim}
%HTTP/1.1 204 OK
%Server: occi-server/x.x OCCI/1.1
%\end{verbatim}
%
%
%\section{HTTP methods applied to collections URLs}
%
%\ralf{NOT fully updated yet!}
%
%This section describes the HTTP methods used to manipulate collections. Each
%HTTP method described is assumed to operate on an URL referring to a collection
%of elements, an URL such as the following:
%\begin{verbatim}
%  http://example.com/storage/
%\end{verbatim}
%
%A collection consist of a set of resource instances and there are three
%different types of collections which may be exposed by an OCCI server.  The
%request and response format is identical for all three types collections
%although the semantics differ slightly for the PUT and POST methods.
%\begin{description}
%\item[Kind locations] The location associated with an OCCI Kind instance
%represents the collection of all resource instances of that particular Kind.
%\item[Mixin locations] The location of an OCCI Mixin instance represents the
%collection of all resource instances associated with that Mixin.
%\item[Arbitrary path] Any path in the URL namespace which is neither a Kind nor
%a Mixin location. A typical example is the root URL e.g.~{\tt
%http://example.com/}. Such a path combines all collections in the sub-tree
%starting at the path. Therefore the root URL is a collection of all resource
%instances available.
%\end{description}
%
%\subsection{GET collection}
%The HTTP GET method retrieves a list of all resource instances in the
%collection. Filtering and pagination information is encoded in the query string
%of the URL.
%
%\subsubsection{Client GET request}
%The query string of the request URL MUST have the following format:
%\begin{verbatim}
%query-string        = ""
%                    | "?" query-parameter *( "&" query-parameter )
%  query-parameter   = attribute-filter
%                    | category-filter
%                    | pagination-marker
%                    | pagination-limit
%  attribute-filter  = "q=" attribute-search *( "+" attribute-search )
%  attribute-search  = 1*( string-urlencoded |
%                          attribute-name "%3D" string-urlencoded )
%  category-filter   = "category=" string-urlencoded
%  pagination-marker = "marker=" UUID
%  pagination-limit  = "limit=" 1*( DIGIT )
%  attribute-name    = attr-component *( "." attr-component )
%  attr-component    = LOALPHA *( LOALPHA | DIGIT | "-" | "_" )
%  string-urlencoded = *( ALPHA | DIGIT | "-" | "_" | "." | "~" | "%" )
%
%\end{verbatim}
%\ralf{FIXME: UUID in ABNF}
%
%\paragraph*{Filtering} A search filter can be applied to categories and attributes
%of resource instances in a collection. An OCCI server SHOULD support filtering.
%The query parameters MUST be URL encoded.
%
%Attribute filters are specified using the {\em q} query parameter.  A filter such
%as {\tt q=ubuntu+inactive} would match all resource instances whose combined
%set of attribute values includes both the word ``ubuntu'' and ``inactive''. It
%is also possible to match on specific attributes by preceding the search term
%with the attribute name and an equal sign, for example {\tt
%occi.core.title\%3Dubuntu+occi.compute.state\%3Dinactive}.
%
%The category filter is specified using the {\em category} query parameter and
%represent a single Kind, Mixin or Action category to be matched. The following
%query would include only resource instances of the Compute type:
%{\tt category=http\%3A\%2F\%2Fschemas.ogf.org\%2Focci\%2Finfrastructure\%23compute}
%
%\paragraph*{Pagination}
%\ralf{FIXME: marker instead of start}
%\label{sec:collection:pagination}
%An OCCI client MAY request that the server only return
%a subset of a collection. This is accomplished using the {\em marker} and
%{\em limit} query parameters.  An OCCI server MUST support pagination.
%
%The {\em marker} parameter specifies the offset into the collection. A value of
%zero, {\tt marker=0} indicates the beginning of the collection.
%%
%The {\em limit} parameter sets the maximum number of elements to include in the
%response. For example {\tt ?marker=...\&limit=10} would indicate the third page
%with a limit of 10 elements per page.
%
%\paragraph*{Example request}
%
%\begin{verbatim}
%GET /storage/?q=ubuntu+server&limit=20 HTTP/1.1
%Host: example.com
%Accept: application/occi-collection+json
%User-Agent: occi-client/x.x OCCI/1.1
%\end{verbatim}
%
%\subsubsection{Server GET response}
%\begin{verbatim}
%HTTP/1.1 200 OK
%Server: occi-server/x.x OCCI/1.1
%Content-Type: application/occi-collection+json; charset=utf-8
%
%{
%  "collection": [
%    {
%      "kind": "..." ,
%      "mixins": [ ... ],
%      "actions": [ ... ],
%      "links": [ ... ],
%      "attributes": { ... },
%    },
%    { ... },
%    { ... }
%  ],
%  "limit": 20,
%  "size": 137,
%  "next": "http://example.com/storage/?q=ubuntu+server&marker=59b50...9b3&limit=20"
%}
%\end{verbatim}
%
%
%\subsection{POST collection}
%The HTTP POST method is used to create/update one or more resource instances in
%a single atomic request. An OCCI server MUST identify existing resource instances
%using the {\tt occi.core.id} attribute.
%
%\subsubsection{Client POST request}
%\begin{verbatim}
%POST /storage/ HTTP/1.1
%Host: example.com
%Accept: application/occi-collection+json
%User-Agent: occi-client/x.x OCCI/1.1
%Content-Type: application/occi-collection+json; charset=utf-8
%
%{
%  "collection": [
%    {
%      "kind": "...",
%      "mixins": [ ... ],
%      "links": [ ... ],
%      "attributes": { ... },
%    },
%    { ... },
%    { ... }
%  ]
%}
%\end{verbatim}
%
%\subsubsection{Server POST response}
%\begin{verbatim}
%HTTP/1.1 204 OK
%Server: occi-server/x.x OCCI/1.1
%\end{verbatim}
%\ralf{Should we support HTTP 200 returning the whole collection? Or maybe just the resource instances created/updated?}
%
%\subsection{POST collection with ``action'' query parameter}
%{\em todo}
%
%\subsection{PUT collection}
%Replace the entire collection with a new one. \ralf{Should we support this?}
%
%\subsection{DELETE collection}
%Delete the entire collection. \ralf{Should we support this?}
%
%\begin{comment}
%\section{More examples}
%The OCCI demo instance of occi-py%
%\footnote{\tt http://github.com/nyren/occi-py}
%running at {\tt http://www.nyren.net/api/} has an early version of the draft
%JSON rendering available.  Feel free to play around with it. However, please
%note the following limitations:
%\begin{itemize}
%\item The Content-Type is {\tt application/json} and not {\tt application/occi+json}
%which would be more appropriate.
%\item It does not support request data in JSON.
%\item Filtering and pagination is not yet supported.
%\end{itemize}
%
%A few example queries using curl:
%\begin{verbatim}
%curl -i -H 'accept: application/json' http://www.nyren.net/api/-/
%curl -i -H 'accept: application/json' http://www.nyren.net/api/link/
%curl -i -X POST -H 'accept: application/json' http://www.nyren.net/api/compute/
%\end{verbatim}
%\end{comment}

\section{Glossary}
\label{sec:glossary}
\begin{tabular}{l|p{12cm}}
Term & Description \\
\hline
\hl{Action} & An OCCI base type. Represent an invocable operation on a \hl{Entity} sub-type instance or collection thereof. \\

\hl{Category} & A type in the OCCI model. The parent type of \hl{Kind}. \\

\hl{Client} & An OCCI client.\\

\hl{Collection} & A set of \hl{Entity} sub-type instances all associated to a particular \hl{Kind} or \hl{Mixin} instance. \\

\hl{Entity} & An OCCI base type. The parent type of \hl{Resource} and \hl{Link}. \\

\hl{Kind} & A type in the OCCI model. A core component of the OCCI classification system. \\

\hl{Link} & An OCCI base type. A \hl{Link} instance associate one \hl{Resource} instance with another. \\

mixin & An instance of the \hl{Mixin} type associated with a {\bf resource
 instance}. The ``mixin'' concept as used by OCCI {\em only} applies to
 instances, never to \hl{Entity} types. \\

\hl{Mixin} & A type in the OCCI model. A core component of the OCCI classification system. \\

\hl{OCCI} & Open Cloud Computing Interface \\

OCCI base type & One of \hl{Entity}, \hl{Resource}, \hl{Link} or \hl{Action}. \\

OGF & Open Grid Forum \\

\hl{Resource} & An OCCI base type. The parent type for all domain-specific resource types. \\

resource instance & An instance of a sub-type of \hl{Entity}. The OCCI
 model defines two sub-types of \hl{Entity}, the \hl{Resource} type and the
 \hl{Link} type. However, the term {\em resource instance} is defined to
 include any instance of a {\em sub-type} of \hl{Resource} or \hl{Link} as
 well. \\

Tag & A \hl{Mixin} instance with no attributes or actions defined. \\

Template & A \hl{Mixin} instance which if associated at resource instantiation
time pre-populate certain attributes. \\

type & One of the types defined by the OCCI model.  The OCCI model types are
 \hl{Category}, \hl{Kind}, \hl{Mixin}, \hl{Action}, \hl{Entity}, \hl{Resource}
 and \hl{Link}. \\

URI & Uniform Resource Identifier \\
URL & Uniform Resource Locator \\
URN & Uniform Resource Name \\
\end{tabular}


%\section{Contributors}
%
We would like to thank the following people who contributed to this
document:

\begin{tabular}{l|p{2in}|p{2in}}
Name & Affiliation & Contact \\
\hline
Michael Behrens & R2AD & behrens.cloud at r2ad.com \\
Mark Carlson & Toshiba & mark at carlson.net \\
Augusto Ciuffoletti & University of Pisa & augusto.ciuffoletti at gmail.com\\
Andy Edmonds & Zhaw & andy at zhaw.ch \\
Sam Johnston & Google & samj at samj.net \\
Gary Mazzaferro & Independent &  garymazzaferro at gmail.com \\
Thijs Metsch & Intel & thijs.metsch at intel.com \\
Ralf Nyrén & Independent & ralf at nyren.net \\
Alexander Papaspyrou & Adesso & alexander at papaspyrou.name \\
Boris Parak & CESNET & parak at cesnet.cz \\
Alexis Richardson & Weaveworks & alexis.richardson at gmail.com \\
Shlomo Swidler & Orchestratus & shlomo.swidler at orchestratus.com \\
Florian Feldhaus & NetApp & florian.feldhaus at gmail.com \\
\end{tabular}

Next to these individual contributions we value the contributions from
the OCCI working group.

% FIXME: Insert an up-to-date table here!

\section{Intellectual Property Statement}
The OGF takes no position regarding the validity or scope of any
intellectual property or other rights that might be claimed to pertain
to the implementation or use of the technology described in this
document or the extent to which any license under such rights might or
might not be available; neither does it represent that it has made any
effort to identify any such rights. Copies of claims of rights made
available for publication and any assurances of licenses to be made
available, or the result of an attempt made to obtain a general
license or permission for the use of such proprietary rights by
implementers or users of this specification can be obtained from the
OGF Secretariat.

The OGF invites any interested party to bring to its attention any
copyrights, patents or patent applications, or other proprietary
rights which may cover technology that may be required to practice
this recommendation. Please address the information to the OGF
Executive Director.


\section{Disclaimer}
This document and the information contained herein is provided on an
``As Is'' basis and the OGF disclaims all warranties, express or
implied, including but not limited to any warranty that the use of the
information herein will not infringe any rights or any implied
warranties of merchantability or fitness for a particular purpose.


\section{Full Copyright Notice}
Copyright \copyright ~Open Grid Forum (2009-2014). All Rights Reserved.

This document and translations of it may be copied and furnished to
others, and derivative works that comment on or otherwise explain it
or assist in its implementation may be prepared, copied, published and
distributed, in whole or in part, without restriction of any kind,
provided that the above copyright notice and this paragraph are
included on all such copies and derivative works. However, this
document itself may not be modified in any way, such as by removing
the copyright notice or references to the OGF or other organizations,
except as needed for the purpose of developing Grid Recommendations in
which case the procedures for copyrights defined in the OGF Document
process must be followed, or as required to translate it into
languages other than English.

The limited permissions granted above are perpetual and will not be
revoked by the OGF or its successors or assignees.


\bibliographystyle{IEEEtran}
\bibliography{references}

\end{document}
