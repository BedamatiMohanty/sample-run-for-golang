\documentclass[10pt,a4paper]{article}
\usepackage[utf8]{inputenc}
\usepackage{fullpage}
\usepackage{graphicx}
\usepackage{fancyhdr}
\usepackage{comment}
\usepackage{occi}
\setlength{\headheight}{13pt}
\pagestyle{fancy}

% default sans-serif
\renewcommand{\familydefault}{\sfdefault}

% no lines for headers and footers
\renewcommand{\headrulewidth}{0pt}
\renewcommand{\footrulewidth}{0pt}

% header
\fancyhf{}
\lhead{GWD-R}
\rhead{\today}

% footer
\lfoot{occi-wg@ogf.org}
\rfoot{\thepage}

% paragraphs need some space...
\setlength{\parindent}{0pt}
\setlength{\parskip}{1ex plus 0.5ex minus 0.2ex}

% some space between header and text...
\headsep 13pt

\setcounter{secnumdepth}{4}

\begin{document}

% header on first page is different
\thispagestyle{empty}

GWD-R \hfill Ralf Nyrén \\
OCCI-WG \\
\rightline {February 4, 2013}\\
\rightline {Updated: \today}

\vspace*{0.5in}

\begin{Large}
\textbf{Open Cloud Computing Interface - RESTful HTTP Protocol}
\end{Large}

\vspace*{0.5in}

\underline{Status of this Document}

%This document provides information to the community regarding the
specification of the Open Cloud Computing Interface. Distribution is
unlimited.

This is a draft proposal of a new document in the Open Cloud Computing
Interface specification suite.

\underline{Copyright Notice}

Copyright \copyright Open Grid Forum (2013-2014). All Rights Reserved.

\underline{Trademarks}

OCCI is a trademark of the Open Grid Forum.

\underline{Abstract}

This document, part of a document series, produced by the OCCI working
group within the Open Grid Forum (OGF), provides a high-level
definition of a Protocol and API. The document is based upon
previously gathered requirements and focuses on the scope of important
capabilities required to support modern service offerings.


\underline{Comments}
\newcommand{\todo}[1]{\textcolor{red}{TODO: #1}}
\begin{itemize}
\item ...
\end{itemize}

\newpage
\tableofcontents
\newpage

\section{Introduction}
%The Open Cloud Computing Interface (OCCI) is a RESTful Protocol and
API for all kinds of management tasks. OCCI was originally initiated
to create a remote management API for IaaS%
\footnote{Infrastructure as a Service}
model-based services, allowing for the development of interoperable tools for
common tasks including deployment, autonomic scaling and monitoring.
%
It has since evolved into an flexible API with a strong focus on
interoperability while still offering a high degree of extensibility. The
current release of the Open Cloud Computing Interface is suitable to serve many
other models in addition to IaaS, including e.g.~PaaS and SaaS.

In order to be modular and extensible the current OCCI specification is
released as a suite of complimentary documents, which together form the complete
specification.
%
The documents are divided into three categories consisting of the OCCI Core,
the OCCI Renderings and the OCCI Extensions.
%
\begin{itemize}
\item The OCCI Core specification consists of a single document defining the
 OCCI Core Model. The OCCI Core Model can be interacted with {\em
 renderings} (including associated behaviours) and expanded through {\em extensions}.
\item The OCCI Rendering specifications consist of multiple documents each
 describing a particular rendering of the OCCI Core Model. Multiple renderings can
 interact with the same instance of the OCCI Core Model and will automatically support
 any additions to the model which follow the extension rules defined in OCCI
 Core.
\item The OCCI Extension specifications consist of multiple documents each
 describing a particular extension of the OCCI Core Model. The extension documents
 describe additions to the OCCI Core Model defined within the OCCI specification
 suite.
\end{itemize}
%
The current specification consist of three documents.
Future releases of OCCI may include additional rendering and extension
specifications. The documents of the current OCCI specification suite are:

\begin{description}
\item[OCCI Core] describes the formal definition of the the OCCI Core Model
\cite{occi:core}.
\item[OCCI HTTP Rendering] defines how to interact with the OCCI Core Model using the
RESTful OCCI API \cite{occi:http_rendering}. The document defines how the OCCI Core Model can
be communicated and thus serialised using the HTTP protocol.
\item[OCCI Infrastructure] contains the definition of the OCCI Infrastructure
extension for the IaaS domain \cite{occi:infrastructure}. The document defines
additional resource types, their attributes and the actions that can be taken
on each resource type.
\end{description}


\section{Notational Conventions}
All these parts and the information within are mandatory for
implementors (unless otherwise specified). The key words "MUST", "MUST
NOT", "REQUIRED", "SHALL", "SHALL NOT", "SHOULD", "SHOULD NOT",
"RECOMMENDED", "MAY", and "OPTIONAL" in this document are to be
interpreted as described in RFC 2119 \cite{rfc2119}.

\textbf{Andy: we need to state that this document as part of the current document set,
supersedes all previous documents.}


The following terms \cite{rfc3986} are used when referring to URI
components:

\begin{verbatim}
 http://example.com:8080/over/there?action=stop#xyz
 \__/   \______________/\_________/ \_________/ \_/
  |           |            |            |        |
scheme     authority       path        query   fragment
\end{verbatim}

\section{OCCI RESTful HTTP Protocol}

This document specifies the OCCI HTTP Protocal, a RESTful protocol for
communication between OCCI Server and OCCI Client. The OCCI HTTP Protocol
support multiple different data formats as payload. Data formats are specified
an separate documents.

{\em TBD: general intro to REST etc}

\section{Namespace}

The OCCI HTTP Protocol maps the OCCI Core model into the URL hierarchy by binding
\hl{Kind} and \hl{Mixin} instances to unique URL paths. Such a URL path is called
the {\em location} of the \hl{Kind} or \hl{Mixin}.
A provider is free to choose the {\em location} as long as it is unique
within the service provider's URL namespace.
For example, the \hl{Kind} instance%
\footnote{\tt http://schemas.ogf.org/occi/infrastructure\#compute}
for the \hl{Compute} type may be bound to {\tt /my/occi/api/compute/}.

A \hl{Kind} instance whose associated type cannot be instantiated MUST NOT be
bound to an URL path. This applies to the \hl{Kind} instance for OCCI Entity
which, according to OCCI Core, cannot be instantiated \cite{occi:core}.

\todo{sub resources here.}

\subsection{Bound and unbound paths}

Since a limited set of URL paths are bound to \hl{Kind} and \hl{Mixin}
instances the URL hierarchy consists of both {\em bound} and {\em unbound}
paths.
A bound URL path is the {\em location} of a \hl{Kind} or \hl{Mixin} collection.

An unbound URL path MAY represent the union of all \hl{Kind} and \hl{Mixin}
collection ``below'' the unbound path.
\todo{FIXME: Should this be a MUST instead?}

\section{Headers and status codes}

\todo{add all other HTTP niftyness from old spec here.}

OCCI Clients and Servers must include a minimum set of mandatory HTTP headers
in each request and response in order to be compliant.
There is also a minimum set of HTTP status codes which must be supported by
an implementation of the OCCI HTTP Protocol.

\subsection{Mandatory HTTP requests headers}

\begin{description}
\item[Accept] An OCCI Client SHOULD specify the media-types the OCCI data
formats it supports in the {\tt Accept} header.

\item[Content-type] If an OCCI Client submits payload in a HTTP request
the OCCI Client MUST specify the media-type of the OCCI data format
in the {\tt Content-type} header.
\end{description}

\subsection{Mandatory HTTP response headers}

\begin{description}
\item[Content-type] An OCCI Server MUST specify the media-type of the OCCI data
format used in a HTTP Response.
\item[Server] An OCCI Server MUST specify the OCCI HTTP Protocol version number.
\end{description}

\subsection{HTTP status codes}

The below list specifies the minimum set of HTTP status codes an OCCI Client MUST
understand. An OCCI Server MAY return other HTTP status codes but the exact client
behaviour in such cases is not specified.

\begin{description}
\item[200]
\item[204]
\item[301]
\item[400]
\item[403]
\item[404]
\item[405]
\end{description}


\section{HTTP methods applied to entity instance URLs}

\todo{Sanity check this section against v1.1...}

This section describes the HTTP methods used to retrieve and manipulate
individual entity instances. An {\em entity instance} refers to an instance
of the OCCI \hl{Resource} type, OCCI \hl{Link} type or a sub-type thereof
\cite{occi:core}.

Each HTTP method described is assumed to operate
on an URL referring to a single element in a collection, an URL such as the
following:
\begin{verbatim}
  http://example.com/compute/012d2b48-c334-47f2-9368-557e75249042
\end{verbatim}


\subsection{GET entity instance}
The HTTP GET method retrieves a representation of a single (existing)
entity instance.

\subsubsection{Client GET request}
The body of the HTTP GET request MUST be empty.
\begin{verbatim}
GET /compute/012d2b48-c334-47f2-9368-557e75249042 HTTP/1.1
Host: example.com
Accept: application/occi+xxx
User-Agent: occi-client/x.x OCCI/1.1
\end{verbatim}

\subsubsection{Server GET response}
% HTTP status: 200
The body of the HTTP GET response MUST contain a representation of the entity
instance.
\begin{verbatim}
HTTP/1.1 200 OK
Server: occi-server/x.x OCCI/1.1
Content-Type: application/occi+xxx; charset=utf-8

...
\end{verbatim}


\subsection{PUT entity instance}
The HTTP PUT method either {\em creates} a new or {\em replaces} an existing
entity instance at the specified URL.
%
The unique identifier of the entity instance (\hl{Entity}.{\tt id}) MUST be
specified in the request URL.
An OCCI Client MAY also specify \hl{Entity}.{\tt id} in the payload however it
MUST be identical to the identifier specified as part of the request URL.

If an entity instance with the specified identifier does not exist the PUT
request allows the OCCI Client to choose the \hl{Entity}.{\tt id} of the new
instance.
An OCCI Server MAY refuse such a request with HTTP status code 405. This would
indicate that the OCCI Server does not allow user defined entity identifiers.

The PUT method MUST be idempotent, i.e. multiple identical PUT requests should
have the same effect as a single request.


\subsubsection{Client PUT request}
The full representation of the entity instance MUST be supplied in the HTTP
body of the request. The request body MUST only include a representation of
a single entity instance.

If the request represent an OCCI \hl{Resource} (as opposed to an OCCI
\hl{Link}) the representation MUST NOT include any \hl{Link} instances
associated with the \hl{Resource} instance.
A server MUST refuse a request including associated \hl{Link}s

Any OCCI \hl{Link}s associated with an existing OCCI \hl{Resource} MUST be left
intact.
\begin{verbatim}
PUT /compute/012d2b48-c334-47f2-9368-557e75249042 HTTP/1.1
Host: example.com
Accept: application/occi-entity+json
User-Agent: occi-client/x.x OCCI/1.1
Content-Type: application/occi-entity+json; charset=utf-8

{
  "kind": "http://schemas.ogf.org/occi/infrastructure#compute",
  "mixins": [ ... ],
  "attributes": { ... }
}
\end{verbatim}

\subsubsection{Server PUT response}
\begin{description}
\item[Content-Type] {\tt application/occi-entity+json}
\end{description}
Upon success an OCCI server MUST return HTTP status code 200 and a complete
JSON representation of the created/replaced entity instance in
single-entity-instance format.
The response MUST be identical%
\footnote{Provided the entity instance was not changed in the meantime.}
to that of a subsequent GET request of the same URL.

\begin{verbatim}
HTTP/1.1 200 OK
Server: occi-server/x.x OCCI/1.1
Content-Type: application/occi-entity+json; charset=utf-8

{
  "kind": "http://schemas.ogf.org/occi/infrastructure#compute",
  "mixins": [ ... ],
  "actions": [ ... ],
  "links": [ ... ],
  "attributes": { ... }
}
\end{verbatim}


\subsection{POST entity instance (action)}
There are two methods to invoke an OCCI Action using the JSON Rendering.
\begin{enumerate}
\item Supply the query parameter ``action'' together with the request. The value
of ``action'' MUST be the {\tt term} of the action \hl{Category}.
\item Specify {\tt application/occi-action+json} in the Content-Type header
and supply a request payload formatted according to section~\ref{sec:format_action}.
In order to specify action attributes this method MUST be used.
\end{enumerate}
An OCCI Client MAY combine the two methods if the ``action'' parameter's value
is equal to the \hl{Category} {\tt term} in the body.

\subsubsection{Client POST action request}
\begin{description}
\item[Accept] {\tt application/occi-entity+json}
\item[Content-Type] {\tt application/occi-action+json}
\end{description}
The example shows the combined method.
\begin{verbatim}
POST /compute/012d2b48-c334-47f2-9368-557e75249042?action=stop HTTP/1.1
Host: example.com
Accept: application/occi-entity+json
User-Agent: occi-client/x.x OCCI/1.1
Content-Type: application/occi-action+json; charset=utf-8

{
  "category": "http://schemas.ogf.org/occi/infrastructure/compute/action#stop",
  "attributes": {
    "method": "graceful"
  }
}
\end{verbatim}

\subsubsection{Server POST action response}
\begin{description}
\item[Content-Type] {\tt application/occi-entity+json}
\end{description}
If the request Accept header contains {\tt application/occi-entity+json} the
server MAY return status code 200 and a full representation of the entity instance.
Otherwise the server MUST return status code 204 and no response payload.
\begin{verbatim}
HTTP/1.1 204 OK
Server: occi-server/x.x OCCI/1.1
\end{verbatim}

\subsection{POST entity instance}
\begin{description}
\item[Content-Type] {\tt application/occi-entity+json}
\end{description}
\todo{This would imply a partial update of the entity instance. While it is
easy to supply only the attributes to be updated the question is if there are any
valid use cases for partial updates using JSON?}

\subsection{DELETE entity instance}
The HTTP DELETE method destroys an entity instance and any OCCI Links
associated with an OCCI Resource.

\subsubsection{Client DELETE request}
\begin{description}
\item[Content-Type] {\tt application/occi-entity+json}
\end{description}
\begin{verbatim}
DELETE /compute/012d2b48-c334-47f2-9368-557e75249042 HTTP/1.1
Host: example.com
Accept: application/occi-entity+json
\end{verbatim}

\subsubsection{Server DELETE response}
\begin{verbatim}
HTTP/1.1 204 OK
Server: occi-server/x.x OCCI/1.1
\end{verbatim}


\section{HTTP methods applied to collections URLs}

\todo{NOT fully updated yet!}

This section describes the HTTP methods used to manipulate collections. Each
HTTP method described is assumed to operate on an URL referring to a collection
of elements, an URL such as the following:
\begin{verbatim}
  http://example.com/storage/
\end{verbatim}

A collection consist of a set of entity instances and there are three
different types of collections which may be exposed by an OCCI server.  The
request and response format is identical for all three types collections
although the semantics differ slightly for the PUT and POST methods.
\begin{description}
\item[Kind locations] The location associated with an OCCI Kind instance
represents the collection of all entity instances of that particular Kind.
\item[Mixin locations] The location of an OCCI Mixin instance represents the
collection of all entity instances associated with that Mixin.
\item[Arbitrary path] Any path in the URL namespace which is neither a Kind nor
a Mixin location. A typical example is the root URL e.g.~{\tt
http://example.com/}. Such a path combines all collections in the sub-tree
starting at the path. Therefore the root URL is a collection of all entity
instances available.
\end{description}

\subsection{GET collection}
The HTTP GET method retrieves a list of all entity instances in the
collection. Filtering and pagination information is encoded in the query string
of the URL.

\subsubsection{Client GET request}
The query string of the request URL MUST have the following format:
\begin{verbatim}
query-string        = ""
                    | "?" query-parameter *( "&" query-parameter )
  query-parameter   = attribute-filter
                    | category-filter
                    | pagination-marker
                    | pagination-limit
  attribute-filter  = "q=" attribute-search *( "+" attribute-search )
  attribute-search  = 1*( string-urlencoded |
                          attribute-name "%3D" string-urlencoded )
  category-filter   = "category=" string-urlencoded
  pagination-marker = "marker=" UUID
  pagination-limit  = "limit=" 1*( DIGIT )
  attribute-name    = attr-component *( "." attr-component )
  attr-component    = LOALPHA *( LOALPHA | DIGIT | "-" | "_" )
  string-urlencoded = *( ALPHA | DIGIT | "-" | "_" | "." | "~" | "%" )

\end{verbatim}
\todo{FIXME: UUID in ABNF}

\paragraph*{Filtering} A search filter can be applied to categories and attributes
of entity instances in a collection. An OCCI server SHOULD support filtering.
The query parameters MUST be URL encoded.

Attribute filters are specified using the {\em q} query parameter.  A filter such
as {\tt q=ubuntu+inactive} would match all entity instances whose combined
set of attribute values includes both the word ``ubuntu'' and ``inactive''. It
is also possible to match on specific attributes by preceding the search term
with the attribute name and an equal sign, for example {\tt
occi.core.title\%3Dubuntu+occi.compute.state\%3Dinactive}.

The category filter is specified using the {\em category} query parameter and
represent a single Kind, Mixin or Action category to be matched. The following
query would include only entity instances of the Compute type:
{\tt category=http\%3A\%2F\%2Fschemas.ogf.org\%2Focci\%2Finfrastructure\%23compute}

\paragraph*{Pagination}
\todo{FIXME: marker instead of start}
\label{sec:collection:pagination}
An OCCI client MAY request that the server only return
a subset of a collection. This is accomplished using the {\em marker} and
{\em limit} query parameters.  An OCCI server MUST support pagination.

The {\em marker} parameter specifies the offset into the collection. A value of
zero, {\tt marker=0} indicates the beginning of the collection.
%
The {\em limit} parameter sets the maximum number of elements to include in the
response. For example {\tt ?marker=...\&limit=10} would indicate the third page
with a limit of 10 elements per page.

\paragraph*{Example request}

\begin{verbatim}
GET /storage/?q=ubuntu+server&limit=20 HTTP/1.1
Host: example.com
Accept: application/occi-collection+json
User-Agent: occi-client/x.x OCCI/1.1
\end{verbatim}

\subsubsection{Server GET response}
\begin{verbatim}
HTTP/1.1 200 OK
Server: occi-server/x.x OCCI/1.1
Content-Type: application/occi-collection+json; charset=utf-8

{
  "collection": [
    {
      "kind": "..." ,
      "mixins": [ ... ],
      "actions": [ ... ],
      "links": [ ... ],
      "attributes": { ... },
    },
    { ... },
    { ... }
  ],
  "limit": 20,
  "size": 137,
  "next": "http://example.com/storage/?q=ubuntu+server&marker=59b50...9b3&limit=20"
}
\end{verbatim}


\subsection{POST collection}
The HTTP POST method is used to create/update one or more entity instances in
a single atomic request. An OCCI server MUST identify existing entity instances
using the {\tt occi.core.id} attribute.

\subsubsection{Client POST request}
\begin{verbatim}
POST /storage/ HTTP/1.1
Host: example.com
Accept: application/occi-collection+json
User-Agent: occi-client/x.x OCCI/1.1
Content-Type: application/occi-collection+json; charset=utf-8

{
  "collection": [
    {
      "kind": "...",
      "mixins": [ ... ],
      "links": [ ... ],
      "attributes": { ... },
    },
    { ... },
    { ... }
  ]
}
\end{verbatim}

\subsubsection{Server POST response}
\begin{verbatim}
HTTP/1.1 204 OK
Server: occi-server/x.x OCCI/1.1
\end{verbatim}
\todo{Should we support HTTP 200 returning the whole collection? Or maybe just the entity instances created/updated?}

\subsection{POST collection with ``action'' query parameter}
{\em todo}

\subsection{PUT collection}
Replace the entire collection with a new one. \todo{Should we support this?}

\subsection{DELETE collection}
Delete the entire collection. \todo{Should we support this?}

\section{HTTP methods applied to QI}

\todo{write this.}

\begin{comment}
\section{More examples}
The OCCI demo instance of occi-py%
\footnote{\tt http://github.com/nyren/occi-py}
running at {\tt http://www.nyren.net/api/} has an early version of the draft
JSON rendering available.  Feel free to play around with it. However, please
note the following limitations:
\begin{itemize}
\item The Content-Type is {\tt application/json} and not {\tt application/occi+json}
which would be more appropriate.
\item It does not support request data in JSON.
\item Filtering and pagination is not yet supported.
\end{itemize}

A few example queries using curl:
\begin{verbatim}
curl -i -H 'accept: application/json' http://www.nyren.net/api/-/
curl -i -H 'accept: application/json' http://www.nyren.net/api/link/
curl -i -X POST -H 'accept: application/json' http://www.nyren.net/api/compute/
\end{verbatim}
\end{comment}

\section{Glossary}
\label{sec:glossary}
\begin{tabular}{l|p{12cm}}
Term & Description \\
\hline
\hl{Action} & An OCCI base type. Represent an invocable operation on a \hl{Entity} sub-type instance or collection thereof. \\

\hl{Category} & A type in the OCCI model. The parent type of \hl{Kind}. \\

\hl{Client} & An OCCI client.\\

\hl{Collection} & A set of \hl{Entity} sub-type instances all associated to a particular \hl{Kind} or \hl{Mixin} instance. \\

\hl{Entity} & An OCCI base type. The parent type of \hl{Resource} and \hl{Link}. \\

\hl{Kind} & A type in the OCCI model. A core component of the OCCI classification system. \\

\hl{Link} & An OCCI base type. A \hl{Link} instance associate one \hl{Resource} instance with another. \\

mixin & An instance of the \hl{Mixin} type associated with a {\bf resource
 instance}. The ``mixin'' concept as used by OCCI {\em only} applies to
 instances, never to \hl{Entity} types. \\

\hl{Mixin} & A type in the OCCI model. A core component of the OCCI classification system. \\

\hl{OCCI} & Open Cloud Computing Interface \\

OCCI base type & One of \hl{Entity}, \hl{Resource}, \hl{Link} or \hl{Action}. \\

OGF & Open Grid Forum \\

\hl{Resource} & An OCCI base type. The parent type for all domain-specific resource types. \\

resource instance & An instance of a sub-type of \hl{Entity}. The OCCI
 model defines two sub-types of \hl{Entity}, the \hl{Resource} type and the
 \hl{Link} type. However, the term {\em resource instance} is defined to
 include any instance of a {\em sub-type} of \hl{Resource} or \hl{Link} as
 well. \\

Tag & A \hl{Mixin} instance with no attributes or actions defined. \\

Template & A \hl{Mixin} instance which if associated at resource instantiation
time pre-populate certain attributes. \\

type & One of the types defined by the OCCI model.  The OCCI model types are
 \hl{Category}, \hl{Kind}, \hl{Mixin}, \hl{Action}, \hl{Entity}, \hl{Resource}
 and \hl{Link}. \\

URI & Uniform Resource Identifier \\
URL & Uniform Resource Locator \\
URN & Uniform Resource Name \\
\end{tabular}


%\section{Contributors}
%
We would like to thank the following people who contributed to this
document:

\begin{tabular}{l|p{2in}|p{2in}}
Name & Affiliation & Contact \\
\hline
Michael Behrens & R2AD & behrens.cloud at r2ad.com \\
Mark Carlson & Toshiba & mark at carlson.net \\
Augusto Ciuffoletti & University of Pisa & augusto.ciuffoletti at gmail.com\\
Andy Edmonds & Zhaw & andy at zhaw.ch \\
Sam Johnston & Google & samj at samj.net \\
Gary Mazzaferro & Independent &  garymazzaferro at gmail.com \\
Thijs Metsch & Intel & thijs.metsch at intel.com \\
Ralf Nyrén & Independent & ralf at nyren.net \\
Alexander Papaspyrou & Adesso & alexander at papaspyrou.name \\
Boris Parak & CESNET & parak at cesnet.cz \\
Alexis Richardson & Weaveworks & alexis.richardson at gmail.com \\
Shlomo Swidler & Orchestratus & shlomo.swidler at orchestratus.com \\
Florian Feldhaus & NetApp & florian.feldhaus at gmail.com \\
\end{tabular}

Next to these individual contributions we value the contributions from
the OCCI working group.

% FIXME: Insert an up-to-date table here!

\section{Intellectual Property Statement}
The OGF takes no position regarding the validity or scope of any
intellectual property or other rights that might be claimed to pertain
to the implementation or use of the technology described in this
document or the extent to which any license under such rights might or
might not be available; neither does it represent that it has made any
effort to identify any such rights. Copies of claims of rights made
available for publication and any assurances of licenses to be made
available, or the result of an attempt made to obtain a general
license or permission for the use of such proprietary rights by
implementers or users of this specification can be obtained from the
OGF Secretariat.

The OGF invites any interested party to bring to its attention any
copyrights, patents or patent applications, or other proprietary
rights which may cover technology that may be required to practice
this recommendation. Please address the information to the OGF
Executive Director.


\section{Disclaimer}
This document and the information contained herein is provided on an
``As Is'' basis and the OGF disclaims all warranties, express or
implied, including but not limited to any warranty that the use of the
information herein will not infringe any rights or any implied
warranties of merchantability or fitness for a particular purpose.


\section{Full Copyright Notice}
Copyright \copyright ~Open Grid Forum (2009-2014). All Rights Reserved.

This document and translations of it may be copied and furnished to
others, and derivative works that comment on or otherwise explain it
or assist in its implementation may be prepared, copied, published and
distributed, in whole or in part, without restriction of any kind,
provided that the above copyright notice and this paragraph are
included on all such copies and derivative works. However, this
document itself may not be modified in any way, such as by removing
the copyright notice or references to the OGF or other organizations,
except as needed for the purpose of developing Grid Recommendations in
which case the procedures for copyrights defined in the OGF Document
process must be followed, or as required to translate it into
languages other than English.

The limited permissions granted above are perpetual and will not be
revoked by the OGF or its successors or assignees.


\bibliographystyle{IEEEtran}
\bibliography{references}

\end{document}
